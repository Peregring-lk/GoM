% Este archivo es parte de la memoria del proyecto fin de carrera
% de Aarón Bueno Villares. Protegida bajo la licencia GFDL.
% Para más información, la licencia completa viene incluida en el
% fichero fdl-1.3.tex

% Copyright (C) 2010 Aarón Bueno Villares

\chapter{Introducción}
\label{chap:introduccion}

Existen muchos videojuegos de estrategia y táctica bélica tanto en el
mercado como en el mundo libre, pero debido al atractivo de los nuevos
avances en diseño 3D y el paradigma del juego en tiempo real, se está
dejando a los juegos por turnos en un segundo plano. Cuando estamos
ante un juego de táctica militar, en general, hay muchísimos factores
a tener en cuenta a cada paso del juego, desde la posición de los
efectivos representativos de tu ejército, hasta el encaramiento de las
unidades, pasando por la situación de los elementos de escenografía o
los posibles flancos libres del enemigo. Todo esto, obviamente, no
puede ser abarcado por un solo jugador humano en tiempo real,
perdiendo gran parte del control sobre la dinámica de su propio
ejército.

Este juego pretende revivir el interés hacia los juegos por turnos
haciendo resaltar todos los aspectos que en una guerra ocurren y
aumentando el atractivo del mismo al darle un contexto medieval y
fantástico. 

\section{Motivación}
Durante mi estancia en bachillerato, conocí \textit{Warhammer Fantasy
  Battles} -abreviadamente, \textit{WF}-, propiedad de la empresa
\emph{Games Workshop}. El paradigma de juego era para
mí tan novedoso como atractivo. Existía una gran comunidad de
jugadores seguidores de este juego a mi alrededor, y además destacaba
fuertemente entre otros juegos de la misma temática (de hecho, era el
único popularmente conocido).

Los precios de miniaturas, pinturas, pinceles, reglamentos y elementos
de escenografía eran altos, pero la calidad, madurez, entretenimiento,
progreso, versatilidad y oportunidades que se ofrecían lo
compensaban.

Con el paso de los años, la compañia orientó el juego a un público
cada vez mas infantil, con el consecuente ataque a los bolsillos de
padres y madres, en vez de a los bolsillos de los jóvenes,
generalmente con menor disponibilidad económica. En resumen, el precio
escapó a mis posibilidades. Las reglas y la forma de juego también se
simplificó e infantilizó. La calidad de las miniaturas también se
redujo (aunque los diseños eran cada vez mas impresionantes). Tampoco
acompañaba el tiempo disponible de dedicación, pues con la edad se
empiezan a asumir otro tipo de responsabilidades. Y no fui el único
que corrió la misma suerte. Hoy en día, ninguno de los jugadores que
conocí continúan manteniendo su hobby, y al igual que yo, guardan sus
miniaturas en cajas y maletines protegidas con algodón.

El siguiente paso obvio fue buscar alguna alternativa digital que
capturara exáctamente la esencia de \textit{WF} para poder seguir
diseñando y ejecutando tácticas  y enfrentándolas a otros jugadores,
con mayor comodidad y menor esfuerzo y coste. Dicha búsqueda resultó
una quimera, pues no existía tal alternativa. Incluso en foros de
videojuegos en algún lugar de internet encontré a gente que también
buscaba y preguntaba lo mismo que yo, desde diversas partes del mundo
(al menos en el mundo de habla hispana).
  
Pasaron dos años desde aquel entonces hasta que me llegó la hora de
decidir que PFC realizar, y, por un afortunado comentario de
\textit{Manuel Palomo Duarte}, acerca del proyecto de otro alumno
sobre otro juego de la misma empresa, el conocido y afamado
\emph{Blood Bowl}, durante la clase de diseño de videojuegos, me di
cuenta que quizás yo debía ser la persona responsable de construir
aquello que con tanta esperanza buscábamos (puesto que era el único de
entre aquellos amigos jugadores de \textit{WF} que empezó a estudiar
informática en la universidad y tenía capacidades suficientes para
crear un videojuego modesto). Y así fue como nació \gomf.

\section{Alcance}
Existen muchos jugadores de juegos de mesa que recrean batallas del
rol de \gomf, obligados por las empresas productoras de estos juego a
comprar un gran número de miniaturas de alto precio para poder
disfrutar de su hobby. Con este proyecto se pretende soslayar estas
dificultades con un producto de calidad que recree estas situaciones
para que estos jugadores puedan competir contra otros y poner a prueba
sus capacidades tácticas, gratuitamente y con un control automático de
las reglas sobre las que se construye el juego. Además al ser el juego
de licencia pública y gratuita, se favorecen las posibilidades de una
rápida expansión y la llegada a muchas manos (se espera que en un
futuro al menos, todo jugador o ex-jugador de \textit{WF} llege a
conocerlo).

Y la criatura se llamará \gomf, y será la culminación final de estas ideas.

\section {Sobre este documento}
Este documento es la memoria de la realización del proyecto fin de
carrera de la titulación \emph{Ingeniería Técnica en Informática de
  Sistemas} (abreviadamente ITIS), de la Universidad de Cádiz, de
\textit{Aarón Bueno Villares} (un servidor).

El documento se organiza en los siguientes capítulos:

\begin{enumerate}
\item \textit{Introducción}: Es el capítulo que actualmente estás
  leyendo. Da una somera introducción a la idea que da título al
  proyecto y a la memoria emergente a partir del mismo. 
\item \textit{Calendario}: Organización temporal de desarrollo del
  proyecto.
\item \textit{Especificación de requisitos del sistema}:
  Especificación y establecimiento de la funcionalidad que ofrecerá el
  software.
\item \textit{Análisis y diseño del sistema}: Descripción del proceso
  de desarrollo del software.
\item \textit{Principales problemas de implementación}: Principales
  problemas de implementación encontrados en la implementación
  efectiva del software.
\item \textit{Pruebas}: Metodología usada para las pruebas de
  verificación del software.
\item \textit{Herramientas}: Herramientas usadas como apoyo en la
  realización del software.
\item \textit{Conclusiones}: Experiencia, resultado, nuevas ideas y en
  general, toda aquella sensación experimentada durante la realización
  del proyecto.
\item \textit{Diversos apéndices}: Manual del usuario, reglamento de
  \gomf, bibliografía, licencia \emph{GFDL} y licencia \emph{GPL}. 
\end{enumerate}
