% Este archivo es parte de la memoria del proyecto fin de carrera
% de Aarón Bueno Villares. Protegida bajo la licencia GFDL.
%
% Para más información, la licencia completa viene incluida en el
% fichero fdl-1.3.tex

% Copyright (C) 2010 Aarón Bueno Villares

\chapter{Conclusiones}
\label{chap:conclusiones}

\gom ha sido mi primer proyecto. Y es evidente que su realización no
me iba a dejar indiferente. No ha sido nada fácil construir una idea
clara sobre lo que me dispondría a hacer, no ha sido nada fácil
poner en práctica dicha idea, así como tampoco ha sido nada fácil
solucionar los problemas del camino. Por último, no ha sido nada fácil
realizar esta memoria.

En total, he aprendido varias cosas:
\begin{enumerate}
\item No hay que dejarse seducir por las mas sabrosas ideas, y mucho
  menos intentar sostenerlas sin premeditación. Quizás hubiera sido
  mas suave la realización de este proyecto si desde el primer momento
  hubiese pretendido crear un software simple e ir haciendolo mas
  complejo de forma progresiva, en vez de empezar por una idea inicial
  compleja que luego tuve que simplificar.

\item He aprendido lo importante que es un buen reparto del trabajo, y
  lo importante que es probar cada parte de forma independiente, en
  vez de intentar conseguir que un sistema integrado funcione de forma
  conjunta.

\item Es crucial ver la diferencia y la naturaleza de un proyecto con
  volumen, frente a un proyecto de pequeña escala como a los que
  hemos estado acostumbrados en el transcurso de los años
  académicos. Sobre la utilidad de la ingeniería del software en este
  aspecto, se podrían decir muchas cosas. Lo que sí es seguro es que
  una metodología tiene un fundamento sólido basado en la experiencia,
  y como dicen nuestras abuelos y abuelas, la experiencia es un
  grado. En un proyecto de corto volumen, se puede abarcar la
  totalidad del problema con sencillez. Papeles en sucio, garabatos y
  esbozos son los máximos representantes de esta clase de
  situaciones. Y \gom no se puede realizar con papeles en sucio,
  garbatos y esbozos. Hace falta el prisma del orden. Y quizás esa sea
  la lección más importante aprendida con la realización de este
  proyecto: las metodologías son orden en la realización de
  proyectos software, y ese orden es necesario cuando el problema
  abarca una conjunto amplio de ideas.
\end{enumerate}

Espero que este proyecto sea respaldado en un futuro, aunque sea
lejano, por un grupo respetable de seguidores aférrimos. Y aun sin
conseguir este propósito, el valor de lo aprendido en esta realización
compensa la posible y no descartable soledad de mi criatura. Entre
tanto, intentaré potenciar mi producto para hacerlo mas extenso,
incrementar su calidad, su riqueza en funcionalidad, así como su
belleza artística. Solo espero no ser el único, y que con el paso del tiempo mas
desarrolladores se animen en mi travesía de construir este videojuego,
que, según alcanza mi documentación, hasta hoy es único en su género.

\section{Mejoras futuras}
\label{sec:mejoras}

A continuación, presento una lista de mis ideas a largo plazo de mejoras por y para el proyecto.

\subsection{\ldots de interfaz}
Me interesa mucho el diseño 3D. Uno de los posibles objetivos de
mejora del producto es pasar la interfaz a tridimensional, y añadirle
animaciones.

Y del mismo modo que en \textit{WF}, los jugadores se compran, modelan
y pintan sus miniaturas, también sería posible y muy atractivo crear
un modo de pintura, de modo que, al diseñar un usuario un ejército,
pueda elegir obtenerlo como una matriz de efectivos y unidades sin
vida ni color, y que el usuario decida qué aspecto darle. Es decir,
potenciar al máximo la personalización de su propio ejército. 

\subsection{\ldots de dominio}
En este tramo es donde me surgen un conjunto mas rico de
ideas. Resultaría muy interesante que \gom fuera inteligente, en el
sentido de que \gomf, mas que un juego, pueda ser un propio jugador al
que poder enfrentarse. Quizás con una red neuronal, algoritmos
evolutivos o diseñando un sistema selectivo se pueda conseguir
implementar dicho juego inteligente, de modo que dicho
\textit{jugador virtual} sea entrenado en cada batalla contra la
máquina por el propio jugador real, y así la máquina, con el paso del
tiempo, se convierta en el alterego del usuario. Incluso se podrían
exportar los perfiles conseguidos y enfrentarlos a otros
\textit{alterego's} de otros usuarios. Sería, además, una bonita y
objetiva forma de ver quien de los dos usuarios es mejor maestro en el
arte de la guerra.

También sería deseable añadir un buen soporte de red, para poder crear
concursos e incluso campañas (con mapas de conquista inclusive) de
forma que distintos jugadores reales intenten competir de diversas
formas con otros jugadores reales, cada uno desde su propio
hogar. Esto último provocaría la atractiva idea de crear una página
web, o un foro, donde se pudiesen hacer quedadas, comentar
estrategias, así como mostrar y comentar partidas previas entre otros
jugadores.

Y, por supuesto, el reto innamovible e incuestionable es la de
enriquecer constantemente la funcionalidad del reglamento, el número
de ejércitos, la diversidad de las unidades, o el tipo y formato de
las reglas.

\subsection{\ldots de datos}
No hay nada mas enriquecedor que el autoexámen. Entonces también, y
por último, tengo en mente crear una buena base de datos que guarde la
información de cada partida completada con éxito, y así generar
estadísticas, detectar buenas y malas jugadas en un turno concreto, o
algún movimiento mal escogido, para que así un jugador pueda evaluar
sus propias participaciones y descubrir sus propios progresos y
avances como maestre táctico.

No sería difícil, en este punto, permitir que el usuario pueda
exportar, en forma de imágenes o documentos, representaciones
esquemáticas de partidas concretas (con gráficos sencillos que
representen cada unidad y cada movimiento, ataques, cargas, etc).

\section{Contacto}
\label{sec:contacto}
Para contactar conmigo, envíe un correo a
\emph{abv150ci@gmail.com}. Tambien puede visitar la página web oficial
del proyecto en \emph{https://forja.rediris.es/projects/gom/}, desde
donde podrá obtener mas información, las fuentes e imágenes usadas en
la realización de \gomf, así como poder estar al tanto de las
novedades producidas en su desarrollo.
