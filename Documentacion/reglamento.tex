% Este archivo es parte de la memoria del proyecto fin de carrera
% de Aarón Bueno Villares. Protegida bajo la licencia GFDL.
%
% Para más información, la licencia completa viene incluida en el
% fichero fdl-1.3.tex

% Copyright (C) 2010 Aarón Bueno Villares

\chapter{Reglamento}
\phantomsection
\label{reglamento}

\begin{center}
{\bf\large Preámbulo}

\begin{minipage}{12cm}
{\small
Este documento describe el reglamento general aplicado en cada uno de
los pasos de la resolución de una partida en \gom, versión 1.0. Así,
aquí se incluyen las reglas aplicadas sobre movimiento, combate, disparos
o magia. Este documento debe tenerse presente siempre y de forma estricta a la hora de organizar un ejército o una táctica, pues ideas que requieran actos no regidos en este reglamento no se podrán hacer efectivos en la partida.}
\end{minipage}
\end{center}

\section*{Introducción}
\label{introduccion}
\gom pretende ser una simulación de escenarios en los que poder llevar
a cabo tácticas militares, con un corte medieval y fantástico. Este reglamento es completo, y contiene
todas las reglas aplicables del juego. Nada de lo expresado fuera de
este documento, será legal en una partida de \gomf, y si en algún lugar relacionado con el proyecto oficial de \gom se afirma algo que contradiga cualquier cosa aquí escrita, siempre tendrá mas validez la versión de este reglamento.

\subsection*{Panorámica general}
\gom permite enfrentar a dos ejércitos en una batalla campal. Al igual
que en las batallas históricas, un ejército está formado por unidades,
y el objetivo es organizar dichas unidades para conseguir la
victoria. Y es por ello que \gom es un juego de táctica militar. El
corte u ambiente del juego es medieval y fantástico, y las unidades y
los ejércitos tienen un comportamiento en el campo de batalla que los
asemeja a las batallas que se protagonizaban en el medievo (que son
herencia del paradigma de guerra de la antiguedad clásica) o como vemos en las guerras de muchas películas de
fantasía, y que son muy distintas a como se llevan a cabo las guerras
en la actualidad.

Es por ello que se desea rescatar este espíritu tomando forma en un
juego orientado a su reproducción en un medio digital, paradigma que,
en el mundo de los videojuegos, no consta de ningún ejemplar. El
reglamento en sí sirve para que cualquier jugador pueda generar
batallas militares incluso en su propio salón, con papeles, o con
miniaturas diseñadas. \gom es independendiente de la finalidad
práctica que se le de a este reglamento.

\subsection*{Mecánica del juego}
El juego está organizado en turnos, y consta de dos ejércitos formados
por unidades. Un jugador efectúa su turno de
juego y luego el oponente juega el suyo, hasta 6 turnos por jugador. A
su vez, cada turno de cada jugador está organizado en tres fases a
resolver en orden: movimiento, combate y disparos.

\paragraph{Confección y despliegue}
Una vez ambos jugadores hayan confeccionado sus propios ejércitos y
hayan decidido una posición inicial para cada una de sus unidades,
desplegarán sus tropas en su zona de despliegue (dependiendo de quién
se haya asignado el rol de jugador 1 y quién de jugador 2) y podrá comenzar la batalla.

\paragraph{Unidades, atributos y características}
Cada unidad tiene una serie de atributos que definen las
particularidades de la naturaleza interna de cada uno de los efectivos
de una unidad. Existen
diversos atributos, como movimiento o fuerza. A su vez una unidad
puede tener una serie de características, que no dependen de la
naturaleza de la unidad, sino generalmente de su equipaje o
complementos de la unidad (o también del tamaño de los efectivos),
estas características son la salvación por armadura, la fuerza de
arma, el alcance de arma, la potencia y el rango de ocupación.

\paragraph{Turnos de juego}
Se realizarán hasta seis turnos alternos por jugador. Cada turno contiene tres fases, que se detallarán a continuación:

\begin{enumerate}
\item \textit{Movimiento}
Como su nombre indica, aquí es dónde se permite \emph{elegir} qué movimientos y cargas pueden efectuar tus unidades. Es la fase con mas reglas y restricciones.

\item \textit{Combate}
Si dos o mas unidades se encuentran en un combate, se resolverán en
esta fase, para cada jugador.

\item \textit{Disparos}
Aquellas unidades que dispongan de armas de proyectiles, podrán
disparar a otras unidades enemigas.

\end{enumerate}

\paragraph{Fin de la partida}
Una vez que finalicen los cinco turnos, la partida se anunciará como
finalizada. Se calcula quien ha sido mejor comandante y la categoría de su victoria.

\subsection*{Vocabulario}
\label{vocabulario}
\subsubsection*{Escala}
Aunque los jugadores podrán (y deberían) aplicar cualquier escala
deseada -siempre en proporción a la aquí indicada-, este reglamento
contabilizara todas sus medidas espaciales en unidades que llamaremos
unidades de terreno o simplemente \textit{u}. \textit{u} podrá ser
tanto un metro, como un pixel, como una pulgada (o si apetece, un
kilómetro). Esto ya recae sobre la decisión de los propios
usuarios. Con respecto a los costes económicos de las unidades, a cada
unidad de coste, o también puntuación, se le designara sencillamente con el calificativo \textit{p}. Así, por ejemplo, una unidad de caballeria ocupa un frente de 10u, un lado de 20u, y un coste de 35p.

\subsubsection*{Ejércitos}
\gom enfrenta a dos ejércitos. Un ejército está formado por unidades,
y una unidad por efectivos. El efectivo será el elemento básico de
cada ejército.

\subsubsection*{Despliegue}
Desplegar es el hecho de elegir una posición inicial para cada una de
las unidades de un ejército. Se le denomina fase o tiempo de
despliegue al hecho de colocar las unidades en el campo de batalla,
según el despliegue elegido.

\subsubsection*{Combatientes}
Existirán dos jugadores por partida, y cada uno comandará uno de los
ejércitos enfrentados. Después de que cada jugador haya elegido su
ejército y confeccionado su despliegue, decidirán quién responderá
como jugador 1, y quién como jugador 2. En este sentido, el ejército
es independiente de la partida, y un jugador podrá confeccionar un
ejército y su correspondiente despliegue de forma aislada, y luego
usarlo para diversas partidas diferentes. Lo que no podrá decidir de
forma aislada será su orden de jugador (jugador 1 o jugador 2) que
dependerá de cada partida y del acuerdo llegado con el oponente. 

\subsubsection*{Escenario}
\gom deberá jugarse sobre una superficie de 1.280u de ancho por 600u
de alto, que será llamado escenario o campo de batalla. El formato del
escenario sera apaisado, es decir, que el eje x del escenario mide los
1.280u de ancho, y el eje y mide los 600u de alto. El eje x se expande
de izquierda a derecha, y el eje y en dirección sur-norte.

El escenario se divide en tres secciones rectángulares paralelas al
eje x. La primera se expande en las primeras 100u del eje formando
un rectángulo de 1.280u por 100u. Se denominará zona de despligue
uno, y será la zona de despliegue correspondiente al primer jugador o
ejército. La segunda se expande desde 100u hasta las 500u en el eje y. Será
la que divida a ambos ejércitos al inicio del combate. La tercera se
expande desde las 500u hasta completar el escenario, en las 600u. Se
denominará zona de despligue dos y será la zona de despliegue
correspondiente al segundo jugador.

\subsubsection*{Turnos y fases}
Una partida se organiza en turnos. Se realizarán 6 turnos completos
para cada jugador, de forma alterna. Es decir, el orden será el
siguiente: \textit{Turno 1 del jugador 1}, \textit{Turno 1 del jugador
  2}, \textit{Turno 2 del jugador 1}, \ldots, \textit{Turno 5 del
  jugador 2}.

Cada turno de cada jugador se divide en tres fases, fase de
movimiento, fase de combate y fase de disparos, y deberán jugarse en ese orden. 

\subsection*{Razas}
Cada ejército será de una raza, y existen dos razas distintas: orcos y
humanos. Cada raza define un conjunto de unidades distinto con un
perfil de atributos propio. El conjunto concreto de unidades definidas
en cada raza vendrá expresado en un apartado especial al final de éste
reglamento.


\section*{Atributos y características}
\label{atributos}
Una batalla está compuesta por una gran diversidad de unidades de distinta naturaleza. En \GoM la naturaleza de cada unidad se modula mediante unos valores llamados atributos y unas características. 

\subsection*{Atributos}
\paragraph{Movimiento (M)}
Es el desplazamiento básico de una unidad en un turno de juego. Si una unidad posee un atributo de movimiento M10, significará que su desplazamiento básico es de 10u.

\paragraph{Habilidad de armas (HA)}
La habilidad que tiene cada efectivo de una misma unidad en combate
cuerpo a cuerpo. Está ponderada de 0 a 10. 

\paragraph{Habilidad de proyectiles (HP)}
La habilidad que tiene cada efectivo en el disparo con armas de proyectiles. Está ponderada de 0 a 10. 

\paragraph{Fuerza (F)}
Este atributo mide la fuerza de cada efectivo de una misma unidad en combate cuerpo a cuerpo. Ponderada de 0 a 10.

\paragraph{Resistencia (R)}
Es la dureza del cada efectivo de una misma unidad en combate cuerpo a cuerpo. Ponderada de 0 a 10.

\paragraph{Ataques (A)}
El valor de este atributo indica cuántos ataques tiene un guerrero en
un solo turno de combate.

\paragraph{Heridas (H)}
Este atributo pondera la dificultad para matar a un efectivo de la
unidad. Es un valor mayor que 0. Si un efectivo recibe un número de
impactos igual a su valor de heridas, se considerará una baja y el
efectivo deberá retirarse del juego.

\paragraph{Iniciativa (I)}
Indica el reflejo y rapidez de cada efectivo de una misma unidad en combate cuerpo a cuerpo. Ponderada de 0 a 10.

\paragraph{Liderazgo (L)}
El liderazgo indica la valentia de una unidad. Ponderado de 0 a 10.

\subsection*{Características}
Por otro lado, existen otros valores llamados
\emph{características}. Estos son la potencia, la salvación por
armadura, la fuerza de arma, alcance del arma y el rango de ocupación.En general estos
atributos no son propios de la naturaleza o el entrenamiento del
guerrero, sino más bien de factores externos como armaduras o el peso
que infiera la unidad en la batalla, como por ejemplo, la fuerza
adicional que da ir montado en un caballo. También se incluye como
característica el coste de una unidad.

\paragraph{Potencia (P)}
La potencia es el peso o valor (en sentido figurado) que tiene cada
efectivo de una misma unidad en batalla. Por ejemplo un efectivo
montado a caballo tiene más potencia que un efectivo de infantería, y
a su vez un orco en jabalí tiene más potencia que un humano a pie. La
potencia de una unidad es la potencia sumada de cada uno de sus
efectivos.

\paragraph{Salvación por Armadura (SA)}
La salvación por armadura de una unidad es la dificultad que imponen
sus armaduras, escudos o cascos a la hora de recibir, cada efectivo de
una misma unidad, un golpe. Ponderado de 0 a 10.

\paragraph{Fuerza de arma (FA)}
Es la fuerza que poseen las armas de proyectiles de una
unidad. Funciona igual que el atributo fuerza, pero solo actúa en la
resolución de disparos. Está ponderado de 0 a 10.

\paragraph{Alcance de arma (AA)}
Es el alcance que poseen las armas de proyectiles de una unidad. Sus
valores toman la misma forma que el atributo movimiento.

\paragraph{Rango de ocupación (RO)}
Cada efectivo tiene un \emph{rango de ocupación}, este rango de
ocupación es el espacio que ocupa la presencia del efectivo en el
campo de batalla (el suelo que pisa), que refleja el espacio ocupado
por el efectivo en el mapa. El rango de ocupación de cada efectivo
tiene siempre forma rectángular, y tanto su ancho como su alto es
múltiplo de 5u.

Por ejemplo, un efectivo de una legión de un ejército humano tiene un rango de ocupación de 10ux10u, y un efectivo de caballería un rango de ocupación de 10ux20u (el primer valor indica el rango de ocupación en el eje x del efectivo, y el segundo el rango de ocupación en su eje y).

\paragraph{Coste en puntos}
Existirá una última característica crucial llamada puntuación de un efectivo. Cada efectivo de una misma unidad tiene un mismo coste en puntos. El coste en puntos de una unidad será la suma del coste en puntos de cada uno de sus efectivos. El coste en puntos de cada efectivo de cada unidad está diseñado de modo que sea coherente y equilibrado con el resto de sus atributos y características, por ejemplo, una mejor HA y un alto liderazgo dará al efectivo un coste en puntos mayor que un efectivo con un valor menor en estos atributos.

\subsection*{Perfil de atributos}
Cada unidad del juego vendrá definido por un perfil de valores para cada una de los atributos y características definidas en el apartado anterior, en forma de tabla.

\subsection*{Chequeo de atributos y características}
\label{chequeo}
La mayoría de los atributos pueden ser puestos a prueba durante la partida. Hay dos tipos de chequeos, los que son puesto en común con el adversario, o los chequeos individuales de los atributos.

\subsubsection*{Chequeo individual}
Los atributos y características sustentas a chequeo individual son todas aquellas que se han indicado como ponderadas de 0 a 10. Esto representa un porcentaje. Cuándo se hace un chequeo individual, se genera un número aleatorio entre 0 a 10, si el número generado es mayor que el valor actual del atributo o característica, se falla el chequeo, en caso contrario se supera. Sin embargo, si el número obtenido es 0, se falla el chequeo sin importar el valor del atributo, y si el valor obtenido es 10, se supera el chequeo automáticamente. Esto significa que una miniatura de I0 solo superará su chequeo de iniciativa cuando el número aleatorio obtenido sea 10. Estas reglas evitan que una unidad con un atributo o característica con el valor máximo o  mínimo siempre falle o supere respectivamente su chequeo individual.

\subsubsection*{Chequeo comparativo}
Los atributos y características sustentas a chequeo comparativo son todas aquellas que se han indicado como ponderadas de 0 a 10. Un chequeo comparativo es el hecho de chequear el valor de un atributo o característica de un efectivo con respecto al valor de un atributo o característica de otro efectivo. Cuando comparas dos atributos o características, siempre hay un atributo o característica agente y un atributo o característica pasiva. Por ejemplo, si un orco ataca a un humano y se debe comparar su F con la R del humano, el atributo agente será F, y el atributo pasivo será R. Lo que se hace en estos casos es restarle al atributo agente, la diferencia entre el atributo agente y pasivo. Por ejemplo, si el ataque fue de F4, y la resistencia del objetivo es R5, la diferencia entre ambos valores es 1, entonces el valor modificado del atributo agente F, será 4-1 = 3. Una vez modificado el valor, se hace un chequeo individual del valor modificado.

\section*{Unidades}
\subsection*{Formación de una unidad}
\label{formacion}
Hemos dicho que una unidad está formada por efectivos. Estos efectivos no se pueden organizar de cualquier forma, tienen que hacerlo de manera que formen una unidad bien formada. La formación de una unidad se llama formación en bloque.

Una formación en bloque puede tener, en principio, cualquier número de
efectivos, que deberán situarse unos junto a otros (los rangos de
ocupación de los distintos efectivos son contiguos). La formación debe
tener siempre forma de cuadro, es decir, posee filas con un mismo
número de efectivos y alineados, como una matriz, salvo la última fila
que puede tener cualquier número de componentes si por número no se
consigue llenar entera. Si se consigue llenar, cualquier nuevo
efectivo añadido a la unidad deberá colocarse en la fila
inmediatamente posterior exáctamente detrás y alineado con un efectivo
de la fila anterior. Los efectivos de la última fila deben estar
siempre lo mas centrados posible. Por ejemplo, una unidad con un ancho
de 5 efectivos, y una última fila de 3, tendrá a sus efectivos de la
última fila colocados en las posiciones 2-3-4 (está centrado puesto
que queda un \textit{efectivo libre} a cada lado).

Por último, si el número de efectivos de la última fila no es adecuado
para conseguir que esté centrado (por ejemplo, que el ancho de la
unidad sea de 5 efectivos, y solo haya 2 en la última fila), se
colocarán mas a la izquierda que a la derecha.  Por ejemplo una unidad
con 21 efectivos y un ancho de fila de 6, tendrá 3 filas de 6
efectivos alineados, y una última fila de 3 efectivos colocados en las
posiciones 2-3-4, en vez de 3-4-5. 

Las unidades también tienen un número mínimo de efectivos y un número
máximo. Estos valores son restricciones de confección de
ejército. Al elegir una unidad en el diseño de su ejército, el número
de efectivos elegidos estará comprendido entre dicho número mínimo y
dicho número máximo \refreg{confeccionejercito}. Evidentemente, en el
transcurso de la partida, el número
mínimo de efectivos se reducirá siempre que sea necesario. Cuando
llege a cero, la unidad habrá sido aniquilada. El número máximo de
efectivos no puede superarse puesto que no existen reglas que aumenten
el número de efectivos en el transcurso de la partida.

El número mínimo de efectivos que debe tener cualquier unidad en la
primera fila (y por simple aplicación de la regla anterior, también el
mínimo de efectivos que debe tener cualquier fila de la unidad; salvo
la última), son 4 efectivos, a no ser que el número mínimo de
efectivos (el mencionado en el párrafo anterior) sea también menos que
cuatro. Si la unidad tiene menos de 4 efectivos, por ejemplo 3,
obviamente no hay nada que hacer y la unidad se queda con un ancho de
fila de 3 unidades. Si se añaden nuevos efectivos a la unidad, se
deben de rellenar, como mínimo antes de crear una nueva fila, tantos
efectivos como queden para completar una fila de 4.

El encaramiento de la unidad será unívoca. Esto es, que todos los
efectivos de una unidad de bloque mirarán en una misma dirección. Esto
se reduce a decir que el encaramiento de la unidad vendrá dada por el
encaramiento de su primera fila. La primera fila es la que siempre
dirige a la unidad completa, y cualquier efectivo en filas secundarias
o posteriores no tienen ningún derecho de mando en ninguno de los
sentidos. Esto significa que a la hora de que la unidad efectúe
cualquier acción, será la primera fila la que mande, y será la
visibilidad de la primera fila la que importe \refreg{visibilidad}.

La primera fila de una unidad se llamará frente, la última
retaguardia, la primera columna flanco izquierdo, y la última columna
flanco derecho. Los efectivos se cuentan de izquierda a derecha. Es
decir, que el primer efectivo será el efectivo situado mas a la
izquierda de la primera fila, y el último efectivo de la unidad será
el situado mas a la derecha de la última fila.

\subsection*{Visibilidad de una unidad}
\label{visibilidad}
La visibilidad de un efectivo dentro de una unidad viene dado por su encaramiento. Su
ángulo de visión será de hasta 60º grados a su izquierda y de hasta
60º hacia su derecha, es decir, un ángulo de visión total de 120º. La
línea de visión izquierda se proyecta desde la esquina superior
izquierda del rango de ocupación del efectivo con un ángulo de
inclinación de 60º respecto a la perpendicular de su
frente. Análogamente, la línea de visión derecha se proyecta desde la
esquina superior derecha del rango de ocupación del efectivo con un
ángulo de inclinación de -60º respecto a la perpendicular de su
frente. La visión total del efectivo será el área encerrada por ambas
líneas de visión. Los efectivos situados en filas posteriores de la
unidad se considera que no ven nada, y la visibilidad de la unidad
será la suma de la visibilidad de cada efectivo de su frente, es
decir, será el área encerrada entre la línea de visión izquierda del
primer efectivo del frente y la línea de visión derecha del último
efectivo del frente de la unidad.

En este punto, el primer requisito para que una unidad vea a otra es
que dicha unidad esté dentro, al menos parcialmente, del área de
visión.

El segundo factor que delimita la visibilidad de un efectivo o unidad
son los obstáculos de un terreno o la presencia de otras
unidades. En este punto, para que una unidad vea a otra, basta con
encontrar un segmento del frente de la unidad, de 5u de tamaño, otro
segmento en una arista de la unidad de destino, también de un tamaño
de 5u, y conectar los extremos formando un cuadrilatero. Si existe al
menos un cuadrilatero de estas características que esté completamente
libre, es decir, que ni siquiera esté parcialmente ocupada por ninguna
otra unidad, la unidad estará viendo a dicha unidad objetivo.

En realidad, cuando un cuadrilatero de estas características está
parcialmente ocupado, no significa que la unidad no pueda ver que hay
``mas allá'', solamente que la unidad no es capaz de identificar qué
está viendo, y en este sentido, el ``ver'' y el ``caracterizar'' se
confunden con este criterio, pues las situaciones en la que es
necesario saber si una unidad ve a otra también es necesario admitir
que la unidad sabe efectivamente que lo que está viendo es una unidad enemiga, ya que
va a realizar una carga o un disparo.

El hecho de elegir ese ``ancho'' de 5u de visión es una medida
arbitraria suficiente para caracterizar la visión de una
unidad. Ampliar o disminuir este ancho, sería equivalente a exigir
un mayor o menor grado de calidad de la visión para caracterizar
cuando una unidad \emph{reconoce} a otra.

\section*{Preámbulos a una batalla}
Antes de comenzar una partida, hace falta elegir una raza, diseñar un
ejército, su despliegue y colocar las tropas en el campo de batalla en
su lugar asignado correspondiente.

\subsection*{Confección de un ejército}
\label{confeccionejercito}
Para confeccionar un ejército se elige una raza, y por último, se
elige un conjunto de unidades con un número determinado de efectivos
en cada una (que podrá variar de una unidad a otra) y la cantidad de
efectivos inicial en el frente de la unidad, según los definidos por
la raza elegida y las restricciones que ésta imponga. El tamaño en
puntos del ejército es libre, en el sentido de que se podrá
confeccionar un ejército tan grande como se desee. Las únicas
restricciones al tamaño del ejército son las impuestas por las
dimensiones del escenario, y en concreto, por la zona de despliegue,
es decir, todo el ejército debe caber en la zona de despligue. Tampoco
se recomienda diseñar un ejército con demasiadas unidades, pues
podría limitar la maniobrabilidad. El tamaño impuesto para un
escenario de \gom es lo suficientemente amplio como para manejar desde
ejércitos pequeños a ejércitos mucho mas modestos.

Además, la confección de un ejército va íntimimamente ligada a la
confección de su despliegue, y estas dos propiedades de un ejército
deben idearse juntas. La confección de un ejército no se da por
concluida hasta que el despliegue del mismo no esté diseñado.

Por último, una unidad tendrá siempre unas restricciones respecto al
número mínimo y número máximo de efectivos con los que la unidad podrá
desplegarse. En la descripción en las listas de ejército se indicará
cuales son estos valores para cada unidad. En la confección del
ejército, no se podrán diseñar unidades con un número de efectivos de
despliegue distinto a este intervalo de valores impuesto.

\subsection*{Confección del despliegue}
\label{despliegue}
Desplegar es posicionar al ejército en el campo de batalla antes de
que ésta comience. Se impone que toda unidad despliegue mirando hacia
el frente, se decir, con un ángulo de 0º respecto al eje x. Esto
provoca la deseable situación de que los dos ejércitos comiencen
frente a frente. A su vez, se impone, evidentemente, que toda unidad
despliegue en su zona de despliegue \refreg{vocabulario}. Como la confección del ejército
es independiente de una partida concreta (los ejércitos serán
\textit{reusables}), no se puede suponer nada acerca de la zona de
despliegue asignada. Es por tanto inadmisible intentar confeccionar un
ejército conforme a una información que se conocerá luego. Por tanto,
confeccionar el despliegue equivale a confeccionar un despliegue con
la suposición añadida de que el jugador será el primero. Si luego
resulta no serlo, solo hace falta desplegar las tropas en su posición
simétrica respecto al campo de batalla.

Por lo tanto, el único dato que se debe especificar al diseñar el
despliegue de cada unidad únicamente será la posición de la unidad en
el campo de batalla, y esto se hará declarándo la coordenada respecto
al eje x e y de la esquina superior izquierda de la unidad; como la
unidad se despliega recta, y el rango de ocupación, número de
efectivos y frente de la unidad es conocido, la posición final de la
unidad es inequívoca. Evidentemente, en el despliegue, toda unidad
debe estar comprendida dentro de su zona de despliegue, al
completo. Esto implica que no podrá colocarse ninguna unidad total ni
parcialmente fuera de dicho espacio. Por ejemplo, una unidad de frente
ancho cuya esquina superior derecha se declare muy cerca del borde
derecho, podrá dejar la parte derecha de la unidad fuera de la zona de
despliegue, y en general, fuera del campo de batalla. Esta es una
situación que ha de controlarse. Unidades con demasiadas filas también
pueden escapar parcialmente de la zona de despliegue.

\subsection*{Organización de una partida}
\label{organizacionpartida}
Una batalla es única y exclusivamente el enfrentamiento de dos
ejércitos. Esto implica un previo acuerdo entre los dos jugadores que
harán de comandantes para ambos ejércitos. Los dos ejércitos,
evidentemente, podrán ser tanto de la misma raza como de razas
distintas. Por otro lado, los dos ejércitos no tienen ninguna
restricción respecto a su valor en puntos, es decir, que podrán
enfrentarse, si así lo desean los jugadores, y sobre todo, si así de
valiente es el primer jugador, ejércitos de 1.000p contra
3.000p. Luego, los jugadores deben ponerse de acuerdo para establecer
quien será el jugador 1, y quién el jugador 2. Despues del despliegue,
el jugador 1 será el que comienze la partida y luego el jugador 2 le seguirá.

\subsection*{Configuración del escenario}
\label{escenario}
La elementos de escenografía a elegir, y su posición en el escenario
es delimitada con las siguientes dos restricciones. La primera es que
los elementos de escenografía no podrán situarse en la zona de
despliegue, y la segunda es que dichos elementos no podrán tampoco
estar a menos de 40u de dichas zonas de despliegue.

Además, el terreno ocupado por los elementos de escenografía se considerará
terreno impasable (es decir, que las unidades no podrás atravesar
dicha zona). Respecto a si dificultará o no la visión de otras unidades,
debería depender de la naturaleza del elemento de escenografía, pero se deja libertad a este efecto. 

Una vez elegido la cantidad y el tipo de los elementos de
escenografía, se colocarán antes del despliegue a través del escenario de forma aleatoria o
decidida, pero siempre respetando las dos restricciones anteriomente impuestas.

\subsection*{Turno de despliegue}
\label{despliegue}
Una vez acordados los ejércitos, comprobada su similitud en puntos, y
su correcta planificación de despliegue, cada jugador coloca sus
unidades correctamente en su zona de despliegue correspondiente, de
modo que toda unidad de cada ejército esté encarada con la línea de
batalla del ejército enemigo, y cuya esquina superior izquierda
coincida con la planificada en el diseño del despliegue.

El jugador segundo se encuentra ante una pequeña dificultad. Tal y
como se comentó en el apartado de confección del despliege, ambos
jugadores debieron suponer que serían el jugador primero. Como
consecuencia, la colocación de sus unidades según sus coordenadas le
llevará a colocarlas en la zona de despliegue del primer jugador. Para
soslayar esta dificultad, solamente hay que desplegar al ejército
segundo en la segunda zona de despliegue a partir de las coordenadas
diseñadas para la zona de despliegue primera, bajo la siguiente
regla: \[(x,y)=(1280 - x, 600 - y)\]

De este modo, se obtendrá la situación simétrica del ejército del
segundo jugador respecto a la zona de despliegue uno. Otra solución
sería considerar momentáneamente a la esquina superior derecha del
escenario como nuevo origen de coordenadas, y desplegar las tropas tal
cual vienen especificadas en la propia lista de ejército. El efecto es
equivalente.

Al proceso de despliegue, y al tiempo transcurrido desde ese momento,
hasta el comienzo del primer turno por el primer jugador, se le
denominará turno de despliegue o turno 0.

\section*{Turnos y fases de juego}
\label{turnos}
Una vez desplegados los ejércitos, puede comenzar la batalla. El
esquema de partida es muy simple, el jugador primero comienza el turno
primero, y realiza su fase de movimiento, y acto seguido, su fase de
combate, para finalizar con su fase de disparo. Una vez terminado, el
jugador segundo comienza su primer turno, y procede de la misma
forma. Luego el jugador primero realiza su segundo turno, y así
sucesivamente hasta que ambos jugadores hayan completado
satisfactoriamente 6 turnos completos de juego. Acto seguido se
calcula la categoría y dirección de la victoria, y finaliza la
partida.

Mas esquemáticamente, un turno de un jugador se divide en el siguiente
juego de fases y subfases:
\begin{itemize}
\item Inicio de turno
\item Fase de movimiento
\begin{itemize}
\item Declaración de cargas
\item Movimiento de cargas
\item Resto de movimientos
\end{itemize}
\item Fase de combate
\begin{itemize}
\item Resolución de combates
\item Efecto de los combates
\end{itemize}
\item Fase de disparo
\end{itemize}

Se describirá cada fase mas concisamente en las secciones que vienen a
continuación.

\subsection*{Inicio de turno}
\label{inicioturno}
El inicio de turno no corresponde una fase por sí misma. Solamente es
un espacio inicial de turno reservado para realizar ciertas acciones
sobre las unidades que huyen \refreg{huidas}.

En el inicio de turno se tiene que comprobar, para cada unidad del
ejército cuyo turno esté en curso, si puede reagruparse. En caso de no
conseguirlo, deberá continuar con su movimiento de huida \refreg{reagrupamiento}. El orden en
que se resuelvan estas huidas no importa, y se realizan en cualquier
orden deseado.

\subsection*{Fase de movimiento}
\label{fasemovimiento}
En esta fase es en donde las unidades que no estén huyendo del
ejército cuyo turno esté en curso podrán moverse libremente. Está
compuesta por las siguientes subfases:

\begin{itemize}
\item Declaración de cargas
\item Movimiento de cargas
\item Resto de movimientos
\end{itemize}

\subsubsection*{Declaración de cargas}
En esta subfase es donde las unidades pueden declarar sus cargas \refreg{dinamica}. Se
declaran en un orden preciso que hay que anotar, ya que de este orden
dependerá muchas de las acciones posteriores \refreg{declaracion}.

\subsubsection*{Movimiento de cargas}
En esta subfase se mueven todas las cargas en el mismo orden en que
fueron delcaradas. Para mover una carga, hay primero que verificar si
la unidad llegará a su objetivo. Luego, si efectivamente se verifica
que la carga será posible, se efectúa la carga \refreg{cargar}. Si no, se efectúa un
movimiento de carga fállida \refreg{cargafallida}.

\subsubsection*{Resto de movimientos}
En esta subfase es donde el usuario puede mover sus unidades, y en
cualquier orden, de forma libre. Aquí las unidades podrán marchar
\refreg{marcha} y realizar maniobras \refreg{maniobras}.

\subsection*{Fase de combate}
\label{fasecombate}
Esta fase está diseñada para resolver los combates existentes en la
partida desde turnos anteriores, o generados en la fase de movimiento
gracias a nuevas cargas \refreg{combate}. Se divide en las siguientes dos subfases:

\begin{itemize}
\item Resolución de combates
\item Efecto de los combates
\end{itemize}

\subsubsection*{Resolución de combates}
La fase de resolución de combates es la fase donde se resuelven todos
los combates existentes en la partida actual.

Los combates deberán resolverse en el mismo orden relativo en el que
fueron creados, según su antigüedad \refreg{ordencombates}. Cada
combate se ejecuta según la iniciativa de cada una de las unidades involucradas, y de la
antigüedad de sus participantes \refreg{resolucion}.

Para cada combate, se anota su resultado según un criterio de
puntuación, que luego se establecerán, nuevamente, en el mismo orden
que los mismos combates fueron resueltos, en la subfase de efecto de
los combates \refreg{resultadocombate}.

\subsubsection*{Efecto de los combates}
En esta subfase se generan los efectos de los combates. Se pueden
identificar tres tipos de efectos: desmoralización, efectos mágicos y
disolución de combates \refreg{resultadocombate}.

La desmoralización es referida a las unidades que han perdido el
combate, con su posible correspondiente huida \refreg{huidas}. Los efectos mágicos,
referidas a los ganadores \refreg{magia}. La disolución de un combate es el hecho de
que ese combate finalice. Si no lo hace, el combate queda vigente
hasta la fase de combate del turno del siguiente jugador.

\subsection*{Fase de disparo}
\label{fasedisparo}
En esta fase las unidades que posean armas de proyectiles podrán
disparar a unidades enemigas \refreg{disparo}. El orden de los disparos es libre, y
para cada unidad que decida disparar, los disparos y sus consecuencias
se ejecutan inmediatamente a continuación. Luego, si se desea, una
nueva unidad podrá disparar, y se efectuarán sus consecuencias antes
de comenzar con una tercera unidad, y así tantas veces como se desee.

\section*{Reglas de movimiento}
\label{movimiento}
En esta sección se presentan todas las reglas referidas al
movimiento. El movimiento no solo está presente en la fase de
movimiento, ya que, por ejemplo, el reagrupamiento pertenece a la
dinámica del movimiento de las unidades en \gomf, sin embargo
este movimiento se efectua en la fase de inicio de turno, y no en la
fase de movimiento. Es por ello que en esta sección se explican las
reglas \emph{generales} de movimiento, aplicables en distintas
situaciones de una batalla.

\subsection*{Movimiento}
Moverse es el hecho de que una unidad cambie de posición en el
escenario (por ejemplo, que pivote) o de estado de movimiento (por
ejemplo, que su movimiento sea de marcha). Los desplazamientos o
cambios de posición siempre tienen un coste, y éste se mide en
unidades de terreno. El movimiento de una unidad viene dado, principalmente, por la
capacidad M del guerrero. A mayor valor de atributo M mas se podrá
mover la unidad en el terreno de juego en un solo turno. 

El efecto de los desplazamientos o cambios de posición en una unidad
es el siguiente. Si una unidad tiene una capacidad de desplazamiento
Mx, y la unidad realiza un desplazamiento de coste u, a la unidad le
quedarán x-u unidades de terreno para el siguiente turno. En la
práctica, se considera que la capacidad dee desplazamiento de una
unidad es un valor continuo en vez de discreto. Así, una unidad podrá
desplazarse, si lo desea, 5.12u para que le resten 4.88u durante el
resto del turno.

Los movimientos de huida no están sujetas a estas restricciones, y
tienen sus propias reglas de desplazamiento \refreg{huidas}. Éstas huidas deben
desplazarse con independencia de lo que le reste de movimiento a la
unidad en un turno.

En general, la unidad se mueve en dirección frontal, es decir, que en
el transcurso de movimiento no se debe variar el ángulo que hay entre
la cara frontal del rango de ocupación y la linea de dirección en que
se esté moviendo, que siempre deberá ser de 90º.

¿Esto significa que las unidades solo pueden mover hacia delante?,
sería descabellada tal absurdez. Para que una unidad pueda cambiar su
encaramiento, o su ángulo de giro, hay ciertas maniobras que lo
permiten. Así, por ejemplo, para que una unidad se quiera mover en una
dirección de 45º con respecto a su posición frontal, deberá antes
hacer un pivotaje de 45º y luego desplazarse frontalmente. Estos
pivotajes y resto de maniobras pueden hacerse tanto como se deseen en
el transcurso del movimiento \refreg{maniobras}.

Y por último, una unidad que se mueva fuera del campo de batalla jamás
volverá y representa el abandono total de la partida, por lo cual la
unidad deberá considerarse baja a todos los efectos.

\subsection*{Terreno}
El terreno es el tipo de suelo que hay en cada parte del escenario. Hay dos tipos de terrenos: terreno abierto, y terreno impasable.

\subsubsection*{Terreno abierto}
Es el terreno común y normal del juego, y las unidades tendrán un movimiento libre regido por las normas habituales de movimiento del juego (las regidas en esta sección del manual).

\subsubsection*{Terreno impasable}
En un terreno impasable, las unidades, directamente, no podrán pasar y
se consideran obstáculos a todos los efectos. Estos terrenos deben ser
bordeados y nunca se podrán atravesar. Los únicos terrenos impasables
corresponden, en concreto, a todos los elementos de escenografía
situados en el escenario.

\subsection*{Espacio entre unidades}
\label{espacio}
Es importante recalcar que a toda unidad debe respetarse el rango de
ocupación de cada uno de esos efectivos. Así, durante el movimiento, y
en general bajo cualquier situación, menos en un combate, toda unidad
debe esquivar a las restantes, porque eso se reduciría a colisionar o traspasar rangos de
ocupación durante el movimiento.

Hace falta mencionar una regla importante que siempre, y para esto no
hay excepción posible, debe respetarse entre unidades de distintos
bandos. Para que dos unidades se consideren trabadas en combate hemos
dicho que antes hay que declararles carga y luego cargar. Por eso
ninguna unidad puede acercarse demasiado a una unidad enemiga, ya que
si llegan a tocarse, deben luchar (es ilógico e irreal que dos
enemigos estén uno frente al otro y no combatan), y esto solo es
posible hacerse si se declara una carga. Así que es norma que dos
unidades enemigas entre sí siempre deban de estar separadas como
mínimo, a no ser que sea para combatir, una distancia de 10u. Para
unificar el criterio, y como dos unidades, por cuestiones de
maniobrabilidad, tampoco pueden estar juntas, se impondrá que dos
unidades amigas también respeten esta distancia.

\subsection*{Maniobras}
\label{maniobras}
Las maniobras son los movimientos de desplazamiento voluntario
característicos de la subfase de resto de movimiento. Existen tres
tipos de maniobras: desplazamiento, pivotaje y giro.

Para poder realizar una maniobra, la unidad debe estar completamente
libre, es decir, no debe estar huyendo ni trabada en combate.

\subsubsection*{Desplazamiento}
\label{desplazamiento}
Desplazamiento es el hecho de que una unidad cambie su posición a base
de mover hacia delante, sin cambiar su ángulo con respecto al eje x
del escenario. Es decir, siguiendo una dirección perpendicular hacia
delante.

El coste de este movimiento es, precisamente, la distancia que hay
entre los frentes de la posición actual de la unidad, y la nueva. Una
unidad nunca podrá desplazarse una distancia mayor que la restante por
su capacidad de movimiento, es decir, según su atributo de movimiento
menos la distancia ya recorrida o gastada por otras maniobras en el
mismo turno.

\subsubsection*{Pivotaje}
\label{pivotaje}
Pivotar una unidad es cambiar su encaramiento. Para pivotar una unidad
hay dos posibilidades, bien pivotarla a la derecha (para mirar al
este), bien pivotarla a la izquierda (para mirar al oeste), todo
depende del eje escogido. Si se coge como eje el pivote superior
izquierdo de la unidad, el pivotaje será izquierdo (se mantiene fijo
ese eje y se mueve la unidad como un compás). Y recíprocamente para el
pivotaje derecho con eje derecho.

Un pivotaje izquierdo consume un movimiento igual al recorrido por el
último efectivo de la linea frontal de la unidad, es decir, el situado
en la esquina superior derecha. Un pivotaje derecho consume un
movimiento igual al recorrido por primer efectivo de la linea frontal
de la unidad, es decir, el situado en la esquina superior
izquierda. En ambos casos, dicho recorrido es igual a la longitud del
arco realizado por el efectivo indicado en cada caso. Esto quiere
decir que si una unidad con M20 y un frente de 5 efectivos mueve 10u,
y se realiza un pivotaje derecho donde el efectivo quinto se desplaza
5u, a la unidad completa le quedarán otros 5u para completar, si lo
desea, su movimiento, bien desplazándose frontalmente, bien con otra
maniobra como un nuevo pivotaje. Cabe destacar que contra mas ancha
sea la unidad (mas efectivos existan en el frente), mas costoso será
su giro para un mismo ángulo, pues se recorrerá un arco mayor.

\subsubsection*{Giro}
\label{giro}
Un giro no desplaza realmente a la unidad, aunque sí que consume
movimiento. Un giro es simplemente cambiar el frente de la unidad a su
retaguarda. Si la unidad miraba al frente (no puede mirar a otro sitio), la unidad
ahora puede girar para que su retaguardia sea su nuevo
frente. Los efectivos no se mueven, solamente se giran.

Al girar la unidad, su última fila será su nuevo frente. Si la última
fila no estaba completa, se completa usando efectivos de la primera
fila, de modo que, al girar la unidad, la nueva última fila (la que
antes era el frente) tenga a sus efectivos en la misma
posición relativa que los efectivos de la última fila antes del
giro. Lo importante es mantener a la formación de la unidad sin modificaciones.

Una unidad solo podrá girar si no está marchando, y una vez girada, no
podrá marchar en lo que queda de turno.

\subsection*{Movimiento de marcha}
\label{marcha}
El movimiento de marcha representa un movimiento a la carrera, que
incrementa temporalmente el atributo M (solamente para ese turno de
movimiento). El movimiento de marcha de una unidad es el doble de su
capacidad de movimiento actual. Además, como ya se ha dicho, una unidad que
haya girado no podrá marchar en su fase de movimiento, o una unidad
que esté marchado no podrá girar en lo que quede del mismo.

A su vez, si la unidad ha marchado, en la fase de disparo tendrá
penalizaciones en sus disparos de la fase de disparo, si es que llega
a realizar alguno.

\subsection*{Dinámica de cargas}
\label{dinamica}
Las cargas son el único mecanismo posible para que varias unidades
puedan entablar combate \refreg{combate}. Poseen dos subfases correspondientes para
declararlas y ejecutarlas en la fase de movimiento, y por cada carga
realizada con éxito se enriquecerá posteriormente la fase de combate.

\subsubsection*{Declaración de cargas}
\label{declaracion}
La declaración de la carga es tan fácil como indicarle a un oponente
que deseas cargar. Solo se puede declarar una carga a una unidad
visible \refreg{visibilidad}. Estas declaraciones se hacen en su subfase de
movimiento correspondiente \refreg{fasemovimiento}.

Antes de declarar carga no se permite medir el terreno de juego y en
\gom no hay ningún medio para hacerlo. Se debe ponderar a ojo si se cree
que la unidad llegará a cargar o si por el contrario fallará la
carga. Una unidad que desea cargar puede declarar carga a una sola
unidad, aunque una unidad enemiga puede ser objetivo de varias
unidades, que se establecerán en declaraciones independientes.

Es importante ser cuidadoso ponderando el éxito de carga de tu unidad,
pues un error de cálculo puede hacer que la carga sea fállida
\refreg{cargafallida} y colocar a tu unidad en una situación poco
ventajosa.

Por último, solo pueden declarar carga aquellas unidades que no estén
combatiendo ni huyendo, y si una unidad declara carga, no podrá
realizar ninguna acción mas durante el resto de su turno, salvo disparar.

\subsubsection*{Movimiento de la carga}
\label{cargar}
En la subfase de movimiento de cargas se intenta establecer, por cada
declaración realizada y respetando el mismo orden, si cada unidad,
gracias a su movimiento, llega a su objetivo. La cantidad de movimiento
disponible para cada unidad que carga es del doble de su atributo
actual M. Dos o mas unidades que declaren una carga contra
una misma unidad, deberán resolver sus movimientos en el mismo orden,
con la posibilidad de que el movimiento de la primera carga (que
intentará maximizar su posición, véase mas abajo), podrá impedir el
éxito de la segunda. Se recomienda que solo se declaren varias cargas
a una misma unidad cuando haya espacio suficiente para todos.

Por ejemplo, que haya tantos \emph{flancos} libres (en sentido general, es
decir, incluyendo frente y retaguardia) como cargas quieran
declararse; es por ello que mas de cuatro cargas contra una misma
unidad es desaconsejable, a no ser que la unidad enemiga sea muy
grande y haya el suficiente espacio.

Por otro lado, es importante resaltar que no se puede elegir el flanco
al que se desea cargar, solamente se declara la unidad a la que
atacar. Para establecer si la unidad llega al objetivo y de qué forma
se sigue el método presentado a continuación.

Solo puede cargarse a los flancos que estén \emph{delante} de la línea
característica de visión \refreg{visibilidad}. De entre esos
flancos, siempre hay que calcular en qué posición se traba mas
efectivos, siempre y cuando se llege a esa posición. Dicho de otro
modo, de entre todas las posiciones a la que la unidad puede cargar,
se elige la que maximice el número de efectivos que se trabarán en
combate \refreg{efectivoscombatientes}. Evidentemente, en ese caso debe
violarse la condición de la separación entre unidades, pues dos
unidades solo están trabadas cuando están en contacto.

A la posición final (el segmento) donde la unidad de carga se ha alineado con la
unidad enemiga la llamaremos posición de carga. Es inmediato que el
tamaño de este segmento es el del frente de la unidad. Para saber si una
unidad \emph{llega} a una posición de carga, se traza una recta desde
el centro del frente de la unidad al centro de la posición de
carga. Si el tamaño de este segmento es menor que el movimiento
disponible de carga, la unidad llega a dicha posición, de lo
contrario, si esa posición de carga es la única posible, la carga será
fállida \refreg{cargafallida}. Si no es la única posible, hay que buscar otra.

Tampoco será válida una posición de carga si el espacio que existe
entre el frente de la unidad y la posición de carga está ocupado,
aunque sea parcialmente. Esto se reduce a considerar un cuadrilatero
formado tras unir los extremos de los segmentos frente y posición de carga, y verificar que dicho espacio
esté completamente libre.

De entre todas las posibles
posiciones de carga, hay que buscar la que trabe en combate el máximo
número de efectivos. Esto se hace sumando el número de efectivos
trabados para cada unidad, y no solo para la unidad que ha
cargado. Por ejemplo, si la unidad objetivo tiene efectivos con un
RO muy ancho en su eje x, pero poco profundo en su eje y, será mas
óptima la carga por su flanco puesto que la suma en bruto de efectivos trabados
será mayor.

\subsubsection*{Carga fállida}
\label{cargafallida}
Si la carga ha sido fállida, el movimiento a realizar se considera un
movimiento obligatorio. Este movimiento se ejecuta en el mismo momento
en que se sepa que no existe ninguna posición de carga efectiva, es
decir, justo antes de validar el movimiento de la siguiente carga,
respetando el orden de las declaraciones.

Cuando una carga es fállida, se debe realizar un movimiento como
máximo igual al tercio del movimiento de carga, de forma rectilinea (es decir, un
movimiento hacia adelante). Si el espacio que la unidad tiene delante
está ocupado, se desplazará hasta la posición máxima sin violar la
separación entre unidades \refreg{espacio}.

\subsection*{Huidas}
\label{huidas}
Huir es el hecho de que una unidad se vea envuelta en el pánico y
decida abandonar el combate. Los movimientos de huida se realizan en
ciertas ocasiones en las que la situación se ha vuelto peligrosa para
la misma. En esta sección se regulan las reglas acerca de qué
situaciones son las que ocasionan ``peligro'' para la unidad, y como
se han de desplazar esas huidas.

\subsubsection*{Unidades que huyen}
\label{unidadhuye}
Las situaciones en las que una unidad está sustenta a huir son las siguientes:

\begin{enumerate}
\item Cuando una unidad comenzó a huir en turnos anteriores y todavía
  no se ha reagrupado \refreg{inicioturno}.
\item Cuado la unidad es desmoralizada en combate. Su movimiento de
  huida se realiza en la subfase de efectos del combate, dentro de la
  fase de combate \refreg{fasecombate}.
\item Cuando la unidad pierde mas de un 25\% de sus efectivos en la
  fase de disparo enemiga \refreg{fasedisparo}.
\end{enumerate}

Las unidades que estén en las dos primeras situaciones, deberán
realizar un chequeo de liderazgo antes de ver si efectivamente se
declaran como unidades que huyen. Si se falla este chequeo, la unidad
realizará su movimiento de huida \refreg{huidas}. A partir
de entonces (tercer caso), en el siguiente \emph{inicio del turno} de su jugador, tendrá que realizar
otro movimiento de huida, a no ser que se reagrupen \refreg{reagrupamiento}.

En el caso de que deban huir por desmoralización en combate, para
realizar el chequeo de liderazgo, existen unos modificadores al
liderazgo antes de realizar el chequeo \refreg{resultadocombate}.

Cuando una unidad huye, hay cierta probabilidad de que sus atributos
sean modificados (negativamente), por efecto de la magia \refreg{magia}.

\subsubsection*{Movimiento de huida}
\label{movimiento de huida}
El movimiento de huida depende básicamente de la capacidad de
movimiento básico de la unidad que huye. Esta cantidad de movimiento
de huida es aleatorio en cada turno de huida. Si la capacidad de
movimiento de una unidad es M, su movimiento de huida en el turno
actual es la media de la suma de tres resultados aleatorios
comprendidos entre la mitad de n y su triple. Por ejemplo, si la
capacidad de movimiento de una unidad es de 10, y está huyendo, tres
resultados aleatorios posibles entre 5 (mitad) y 30 (triple) son 7,
16, 22, y la media de esas tres cantidades es 15, así que ese turno
moverá 15. Esto pondera el hecho de que en un movimiento de huida,
este es desorganizado, sin formación y sustento a tropiezos, caídas,
cansancio (que pueden disminuir la velocidad de la huida), o pánico
excesivo (que pueden aumentarla), además fomenta el hecho de que la
huida no es controlable por el jugador.

En un movimiento de huida, la unidad que huye intentará por todos los
medios escapar de la situación que le produce peligro. La dirección
del movimiento de huida siempre es en el sentido opuesto a la unidad
que le produce peligro. Si debe huir por motivo de los disparos,
huirá en dirección contraria a dicha unidad. Si es por motivos de un
combate, huirá de la unidad de mayor potencia de entre todas las
unidades enemigas que participen en el mismo combate. Si la huida no
ha sido recién lanzada (es decir, que es del tercer tipo de entre los
anunciados en el apartado anterior), el movimiento de huida continuará
la misma dirección que en los turnos anteriores. En la siguiente
sección se explica el método usado para elegir la dirección y el
comienzo del movimiento de huida.

Si en el movimiento de huida se encuentra a otra unidad, pasará por
encima suya sin modificar su dirección. Si el valor obtenido en su movimiento de huida no
es lo suficientemente alto como para saltar completamente a la unidad,
se otorga la bonificación necesaria para traspasarlo. Si justo detrás
de esa unidad existen otras unidades, se otorga la bonificación
necesaria al movimiento hasta encontrar la primera posición, en la
misma dirección de huida, donde la unidad quepa completamente sin
violar las restricciones de cercanía a otras unidades (10u). Si el
elemento es de escenografía, se aplican los mismos criterios.

Si la unidad por la que ha pasado es amiga, ésta unidad realizará un
chequeo de liderazgo. Si lo falla, deberá huir en dirección contraria
a la unidad que le provoca el pánico, es decir, en dirección contraria
a dicha unidad. Si es enemiga, la unidad que huye perderá la mitad de
los efectivos de su unidad.

\subsubsection*{Dirección del movimiento de huida}
\label{direccionhuida}
Para establecer la dirección del movimiento de huida en las
situaciones de lanzamiento del pánico (situaciones 1 y 2), primero se
considera que la última fila de ambas unidades está completa, para
obtener un rectángulo. Luego se dibuja una recta que se diriga desde
el centro del rectángulo \emph{enemigo} al centro del rectángulo de la
unidad que va a huir. La dirección de esta recta será la dirección de huida.

Luego la unidad de huida gira sobre su centro hasta que su frente sea
perpendicular a dicha recta imaginaria; y de las dos posiciones en las
que esto es posible (podemos girar a la derecha hasta conseguir, o
girar a la izquierda hasta conseguirlo), elegimos la tenga la
retaguardia de la unidad que huye mas cerca del enemigo. De este modo,
al desplazarse la unidad que huye linealmente, se consigue que lo haga alejándose del
enemigo y a su vez respetando la dirección de huida.

\subsection*{Reagrupamiento}
\label{reagrupamiento}
Reagruparse es el hecho de que una unidad detenga su movimiento de
huida y se disponga a participar de nuevo en el combate. El
reagrupamiento se calcula solamente al \emph{inicio del turno}, para
todas las unidades que estén huyendo del jugador cuyo turno está en
curso \refreg{inicioturno}.

Para ver si una unidad se reagrupa, basta con hacer un chequeo
individual del liderazgo de la unidad. Si se supera el chequeo, la
unidad se reagrupara, y ese turno no huirá, quedándose en la misma
posición en la que estaba, pero ahora pudiendose declarar cargar,
efectuar maniobras o ejecutar disparos.

Si el chequeo es fallado, se procede a ejecutar un nuevo movimiento de
huida junto a sus efectos mágicos, tal y como indican las reglas presentadas arriba.


\section*{Reglas de combate}
\label{combate}
En esta fase es donde se resuelven los combates de las unidades que
estén trabadas en combate. Un combate es una agrupación de unidades
en contacto, aunque no necesariamente todas entre sí. Por ejemplo, una
serie de unidades trabadas en combate una detrás de otra, en linea, es
un solo combate, aunque la última unidad no esté combatiendo con la
primera.

\subsection*{Orden de los combates}
\label{ordencombates}
Los combates se resuelven por orden de antigüedad. El combate que más
turnos lleve es el combate que primero se resuelve. Todo combate que
se haya creado en un mismo turno (por ejemplo, porque ha sido en este turno
donde se ha declarado y efectuado la carga), se resuelve por orden
de declaración de la carga, es decir, primero se resuelve el combate
que sea consecuencia de una declaración mas antigua. Si dos combates
son igual de antiguos, se resolverán en el mismo orden en que fueron
resueltos el turno en que comenzaron.

\subsection*{Resolución de un combate}
\label{resolucion}
Una vez reconocido el orden de los combates, se procede a la
resolución en orden de cada uno de ellos. El orden de la resolución es
el siguiente:
\begin{enumerate}
\item Atacan las unidades que hayan entrado al combate en este turno,
  es decir, que hayan cargado en este turno, en el mismo orden en que
  cargaron \refreg{fasemovimiento}.
\item Entre las unidades que lleven mas de un turno combatiendo, sus
  ataques se resuelven por orden de iniciativa. Si dos unidades tienen
  una misma iniciativa, efectúa sus ataques primero la unidad que
  efectúo su carga primero (es decir, el que lleve mas turnos en el
  combate, o si entraron en el mismo turno, el que declaró su carga primero).
\item Se calcula el resultado final del combate \refreg{resultadocombate}.
\item El perdedor se desmoralizará, y acto seguido las unidades
  perdedoras en el combate realizarán un chequeo de liderazgo en el
  mismo orden en que atacaron.
\item Las unidades que fallen dicho chequeo huirán en la subfase de
  efectos de los combates \refreg{fasecombate}, en el mismo orden en que fallaron su
  chequeo.
\end{enumerate}

\subsection*{Efectivos combatientes}
\label{efectivoscombatientes}
Que dos unidades estén trabadas no significa que todos sus efectivos
lo estén. Siempre existe una linea de batalla y en ésta es donde se
define quién puede y quién no puede combatir. En general, para que dos
efectivos se consideren trabados en combate, sus rangos de ocupación
deben ser tangentes entre sí, es decir, que estén chocándose
literalmente los unos con los otros. Una unidad estará trabada a otra
incluso si su único punto de contacto es vértice con vértice entre sus
rangos de ocupación.

\subsection*{Ataques de los bandos}
\label{ataques}
A la hora de declarar los ataques, hay que asignarlos entre todas las
unidades contra la que cada efectivo esté trabado en combate cuerpo a
cuerpo. Si un efectivo solo está en contacto con una sola unidad
enemiga, todos sus ataques se asignarán automáticamente a ella. Si hay
mas de una, el usuario debe decidir como repartir los ataques de cada
efectivo para cada unidad enemiga con la que esté en contacto.

Para resolver un ataque, se procede de la siguiente forma: se realiza
un chequeo comparativo del HA de los atacantes. El HA del agresor será
el atributo activo, y el del agredido, el atributo pasivo. Luego, se
procede de la misma forma con los atributos fuerza del agresor, y
resistencia del efectivo. Por último, se hace un chequeo de SA del
efectivo defensor. El atributo SA es modificado en tantos puntos como
supere la fuerza del atacante a cuatro. Es decir, si la fuerza del
atacante es 6, y la SA por armadura del defensor es 4, el chequeo de
SA se realiza considerando un SA de 4 - (6 - 4) = 2.

La unidad enemiga pierde tantas heridas como ataques realizados con
éxito. Estas heridas, al igual que en la fase de disparo, se reparten
por orden entre sus efectivos, y no de forma repartida. Es decir, si
una unidad tiene 20 efectivos, y su atributo H es 2, se deberá
eliminar un efectivo cada vez que la unidad contabilice dos heridas
menos. No se puede quitar una sola herida a distintos efectivos para
hacer que la unidad pierda sus heridas de modo que no pierda
efectivos, y de ahí la mención al \emph{retiro en orden} de los
efectivos, según las heridas perdidas en la unidad.

Si en un combate, el número de heridas totales es mayor que el número
de heridas de la unidad, a parte de que habrá que eliminar a todos los
efectivos, las heridas restantes siguen contabilizando para el
resultado del combate final. Hace falta recordar para esto que, aunque una unidad haya
desaparecido, pueden existen mas unidades en el mismo combate. Si el
combate era formado solo por dos unidades, la atacante y la agresora,
y una de las unidades desaparece, evidentemente no habrá que
contabilizar estas heridas adicionales porque ya no hay ningún combate
que resolver.

Si una unidad pertenece a un combate, pero solo estaba trabado con una
de las unidades del mismo, y esta unidad es aniquilada, la unidad ya
no pertenecerá al combate, aunque habrá que esperar a que este combate
se resuelva y se midan y ejecuten sus efectos para desligar a la
dicha unidad del mismo.

Luego toca la respuesta del enemigo. Obviamente, si se realizan 6
heridas, y el frente enemigo tiene solo 5, todos han muerto en este
turno, e incluso ha muerto uno de la segunda fila. Como solamente la
primera fila responde (por ser los únicos que estaban trabados en
combate al inicio del combate), la unidad enemiga se quedaría sin
atacar. Es decir, que el número de bajas que haya recibido la unidad
se considera que se eliminan del frente de batalla (porque aquí es
donde las unidades están trabadas) aunque luego en la partida se
eliminen de la última fila de la unidad.

Una vez visto cuantos efectivos de la unidad enemiga pueden atacar, se
resuelven los ataques del mismo modo en que se resolvieron los ataques
de la unidad que atacó primero, y se contabilizan todas las heridas
producidas, y se retiran las bajas convenientes. Y continúa el combate
por la siguiente unidad siguiendo el orden impuesto de resolución de combates.

\subsection*{Resultado del combate}
\label{resultadocombate}
El resultado del combate se calcula de la siguiente forma:
\begin{enumerate}
\item Para cada bando se anotan las heridas realizadas.
\item Para cada bando, se anotan el número de filas de cada una de las
  unidades presentes en el combate.
\item Para cada bando se anotan el número de unidades situadas por alguno de los flancos de cualquier unidad enemiga presente en el combate.
\item Se anota y se multiplica por dos el número de unidades situadas en la retaguardia de cualquier unidad enemiga presente en el combate.
\item Recibe un punto adicional el bando con más potencia total sumada de las unidades implicadas en el combate.
\end{enumerate}

El jugador que haya acumulado mas puntos mediante estas reglas se
considerará ganador del combate. El bando perdedor deberá realizar un
chequeo de desmoralización por cada unidad participante en el combate,
con una penalización al liderazgo igual al número de puntos por el que
ha perdido el combate. Si sucede un empate, no se realiza tal chequeo
y el combate continúa en el siguiente turno de juego. Si algunas
unidades fallan el chequeo, se anotan estas huidas y continúa la
resolución del siguiente combate. Cuando se resuelvan todas, se pasa a
la subfase de efectos de los combates, y las unidades huyen en el
mismo orden en que dichas huidas se calcularon. Si todas las unidades de un
mismo bando en un combate han huido o han sido aniquilidas, la batalla se
dará por finalizada.

Las unidades combatientes podrán ver incrementado los valores en sus
atributos tal y como describen las reglas referentes a la magia (véase
sección correspondiente).

\section*{Reglas de disparo}
\label{disparo}
La fase de disparo es la fase en la que las unidades que posean armas
de proyectiles podrán disparar.

Al comentar la fase, el jugador en turno puede elegir disparar a una
unidad enemiga. La única restricción es que la unidad objetivo esté
visible. Cualquier unidad que posea un arma de proyectiles, y además un HP superior a 0,
podrá disparar.

El proceso es el siguiente:
\begin{enumerate}
\item El usuario elige una unidad que posea armas de proyectiles y
  tenga un hp superior a 0.
\item Luego, se elige a qué unidad desea disparar, con la única
  condición de que esta unidad esté visible \refreg{visibilidad} y no
  esté combatiendo.
\item La unidad dispara, según las reglas mencionadas a continuación.
\item Se efectúa acto seguido los posibles efectos de los disparos.
\item Si el jugador en turno lo desea, puede elegir otra unidad para disparar.
\end{enumerate}

\subsection*{Armas de proyectiles}
\label{armas}
Las armas de proyectiles no son enunciadas explícitamente entre los
atributos y características del juego. Solo tenemos de ellas la fuerza
del arma y el alcance de los proyectiles. Se considerará que una
unidad posee armas de proyectiles si su FA y su AA es superior a 0.

Es absurdo que exista una unidad cuyo perfil de atributos tenga una FA
superior a 0 y un AA 0, o viceversa. Pero debido a los efectos de la
magia que pueden modificar cualquier atributo, la fuerza del arma,
por ejemplo, puede reducirse a 0. O una unidad que no posee armas
puede incrementar su FA o AA. En estos casos, si la suerte acompaña a
las unidades y su FA y también su AA se hace mayor que cero, se
considerará que la unidad a adquirido un arma.

Evidentemente, una unidad que no está diseñada para disparar, lo mas
probable es que posea un HP 0, por lo que, aunque obtenga un arma, no
sabrá dispararla. Necesitará, por tanto, que los efectos mágicos le
incrementen su HP a 1, además de su FA y AA (que inicialmente será 0),
y entonces la unidad pasará a conseguir proyectiles.

\subsection*{Resolución de los disparos}
\label{resoluciondisparos}
Una unidad, para poder disparar, necesita ver a su objetivo. Para ello
se siguen las reglas de visibilidad entre unidades \refreg{visibilidad}. Además, una vez
declarado un objetivo, para que el disparo sea efectivo, hace falta
que la distancia que separa los centros de ambas unidades no sea mayor
al alcance del arma.

Para calcular la distancia que separa a dos unidades se procede como
habitualmente. Se considera que sus últimas filas están completas, se
toman los centros de ambas unidades, y luego se mira la distancia
del segmento que conectaría ambos centros. Si este tamaño es mayor que
el alcance del arma, el disparo no se hará efectivo y no ocurrirá
nada.

Si el disparo sí que es efectivo, se resuelven los ataques. Se hace de
forma parecida a como se resolvían los combates. En concreto, de la
siguiente forma:

\begin{itemize}
\item Se hace un chequeo individual del HP de la unidad atacante.
\item Para cada disparo que supere el primer chequeo, se realiza un
  chequeo comparativo entre la fuerza del arma y la
  resistencia de la unidad objetivo.
\item Se hace un chequeo individual de la SA por la unidad objetivo,
  para cada disparo que supere el chequeo anterior.
\item Se generan tantas heridas en la unidad objetivo como disparos
  superen el último chequeo.
\end{itemize}

A la hora de hacer estos chequeos, existen ciertas
modificaciones a los atributos implicados:
\begin{itemize}
\item Si el objetivo se encuentra a más de la mitad del alcance del
  arma, se reduce el HP y la FA en una unidad.
\item Si el objetivo se encuentra a menos de la mitad de un cuarto del
  alcance del arma, se incrementa el HP y la FA en una unidad.
\item Si la unidad ha marchado en su turno de movimiento, se reduce el
  HP en una unidad.
\item Si la fuerza del arma es mayor que 4, la SA del enemigo se
  reduce en tantos puntos como FA supere a 4 (al igual que en la fase
  de combate).
\end{itemize}

\subsection*{Efectos de los disparos}
\label{efectosdisparos}
Una vez se resuelva el número de heridas realizadas en los disparos,
se deben calcular sus efectos.

Si el número de bajas realizadas (de efectivos retirados) es mayor al 25\% del número de
efectivos presentes en la unidad objetivo antes de los disparos, la
unidad objetivo de los disparos debe realizar un chequeo de
liderazgo. Si lo falla, deberá huir inmediatamente y antes de elegir
la siguiente unidad a disparar, o antes de realizar cualquier otra
acción, en dirección contraria a la unidad
que ha disparado \refreg{huidas}. Si la unidad objetivo de los
disparos ya estaba huyendo, no huirá
de nuevo y omitirá el chequeo de liderazgo.

Además, si el número de bajas realizas es, como ya hemos dicho, mayor
al 25\% de los efectivos de la unidad, la unidad atacante recibirá
bonificación mágica a sus atributos, como se expone a continuación.

\section*{Reglas de magia}
\label{magia}
La magia en \gom es exclusivamente referida a la modificación de
atributos. Estos atributos son modificados en tres situaciones
diferentes: durante un movimiento de huida (no importa si antes o
después), tras la victoria en un
combate, y tras efectuar mas de un 25\% de bajas a una unidad a la que
se ha disparado.

En cada una de estas tres situaciones se procede de la misma forma:
primero se escoge un atributo o característica de forma aleatoria,
salvo la puntuación y el rango de ocupación, que no serán
modificables.

Luego, se escoge un número de 0 a 3 que pondera la probabilidad de que
ese atributo cambie. Si la probabilidad es 0, el atributo no será
modificado. Si es 1 o 2, se modificará un punto. Si es 4, se
modificará dos. En realidad, el hecho de escoger un número entre 0 y 3
no es importante, lo que importa es que al hecho de no modificar el
atributo le corresponda una probabilidad del 25\%, al hecho de
modificarlo un solo punto el 50 (el más probable), y al hecho de
modificarlo en dos puntos, el 25\% restante. Se puede escoger
cualquier método que respete esta probabilidad.

Si los atributos elegidos son la capacidad de movimiento o el alcance,
el atributo será modificado en la cantidad de punto obtenido en el
paso anterior, pero multiplicando este valor por 10. Eso significa que
si hemos de modificar el movimiento en un punto, lo que haremos es
incrementar M en 10u. Si hemos de modificarlo en 2, lo incrementaremos
20u. Para el caso del alcance del arma se procede de la misma forma.

En el caso de que la unidad esté huyendo, estos efectos son
penalizadores en vez de bonificadores, es decir, que el valor obtenido
es la cantidad de puntos que deben restarse al atributo o
característica elegida aleatoriamente.

Esta clase de efectos mágicos ilustran a la magia como una entidad
presente en el campo de batalla que provoca efectos deseosos en las
unidades fuertes y valientes, y efectos perniciosos en las unidades
débiles y cobardes. Y como ya hemos dicho, incluso pueden otorgar
\emph{armas de proyectiles} a las unidades si estas han sido lo
suficientemente buenas durante el transcurso de la batalla.

\section*{Fin de la partida}
\label{finpartida}
\subsection*{Obtención de puntos}
\label{puntos}
Una vez que la partida acaba, tras seis turnos de juego, se contabiliza el resultado final para evaluar quién gana y quién pierde. Los puntos a obtener son los siguientes:
\begin{enumerate}
\item Cada jugador obtiene tantos puntos como puntos valgan todas las
  unidades de su ejército que aún estén presentes en el escenario en
  el campo de batalla, que no estén huyendo, y además que su potencia
  sea superior a 5.
\item Si una unidad ha perdido mas de la mitad de sus efectivos
  iniciales al comienzo de la batalla, se contabilizan solo la mitad de sus puntos.
\item Se divide el tablero en 4 cuadrantes iguales. Por cada
  cuadrante, se contabiliza la potencia total de unidades presentes en
  ese cuadrante, para cada bando. Por cada cuadrante, el ejército que
  tenga mas potencia de unidad en dicho cuadrante, obtiene 100 puntos
  adicionales.
\end{enumerate}

\subsection*{Tipos de victorias}
\label{victorias}
No siempre se gana de la misma forma, hay veces que se puede haber
ganado por una diferencia justa, o haber ganado
abrumadoramente. Primero, al bando que haya obtenido un número mayor
de puntos se le resta el que haya obtenido un número menor, y a partir
de este valor (llamado valor de victoria) se pueden obtener 4 tipos
distintos de resultados en la partida:
\subsubsection*{Masacre}
Una victoria será masacre cuando el valor de victoria sea mayor o
igual al
doble de los puntos contabilizados al ejército perdedor. Es decir, que
si los dos ejércitos han conseguido respectivamente 460 y 100 puntos,
el valor de victoria será de 360, y 360 es un valor mayor que el doble
de 100. El primer ejército, por tanto, habrá masacrado al segundo.

\subsubsection*{Victoria decisiva}
Una victoria será decisiva cuando el valor de victoria sea mayor o
igual a los puntos conseguidos por el ejército perdedor, sin llegar a
ser masacre. Por ejemplo, si los dos ejércitos han conseguido
respectivamente 460 y 200, la victoria será decisiva, porque el valor
de victoria será de 260, y 260 es mayor que 200 pero no mayor a su
doble, que es 400.

\subsubsection*{Victoria marginal}
Una victoria será marginal cuando el valor de victoria sea mayor o
igual a la mitad de los puntos conseguidos por el ejército perdedor,
sin llegar a ser victoria decisiva. Por ejemplo, si los dos ejércitos
han conseguido respectivamente 460 y 300, la victoria será marginal,
porque el valor de victoria es de 160 puntos, y 160 es mayor que 150,
que es su mitad, pero no mayor que 300, que es su valor en puntos de
batalla, y es por ello que no llega a ser victoria decisiva (aunque solo por
una diferencia de 10 puntos).

\subsubsection*{Empate}
Una partida queda empatada si no estamos en ninguna de las situaciones
anteriores. Es decir, cuando el valor de victoria sea 0, o el ganador
tenga un valor de victoria menor que la mitad que la puntuación del
perdedor. Por ejemplo, si los dos ejércitos han conseguido
respectivamente 460 y 360 puntos, el valor de victoria será de 100
puntos para el primer ejército, y la partida se considerará acabada en
empate, ya que 100 es menor que 180, que es el valor en puntos del
ejército ``perdedor''.

\section*{Listas de ejército}
En esta sección se incluirán los perfiles de atributos que configuran
a las nueve unidades de ambas razas: humanos y orcos.

Los perfiles de atributos se muestran en forma de tabla nemónica, cuya
leyenda es la siguiente:
\begin{description}
\item[Nombre] Aparece a la izquierda de la cabecera de la tabla.
\item[Coste en puntos] Es el valor que acompaña al nombre de la
  unidad, seguida de dos puntos.
\item[RO] Es el rango de ocupación de la unidad. Tiene la
  forma $(x, y)$, donde la $x$ representa su rango de ocupación en la
  coordenada $x$, e $y$ su correpondiente en la coordenada $y$.
\item[M] Capacidad de movimiento.
\item[HA] Habilidad de armas.
\item[HP] Habilidad de proyectiles.
\item[F] Fuerza.
\item[R] Resistencia.
\item[A] Ataques.
\item[H] Heridas.
\item[I] Iniciativa.
\item[L] Liderazgo.
\item[P] Potencia.
\item[SA] Salvación por armadura.
\item[FA] Fuerza del arma.
\item[AA] Alcance del arma.
\item[Nmin] Número mínimo de efectivos para esa unidad, en el diseño
  del ejército.
\item[Nmax] Número máximo de efectivos para esa unidad, en el diseño
  del ejército.
\end{description}

\subsection*{Humanos}
El ejército humano se caracteriza por su firmeza y carácter puro en la
guerra. La cobardía y el desorden no se encuentran entre sus
atributos, y su rol está basado en la disciplina.

\subsubsection*{Legión}
La legión es la unidad mas básica de un ejército humano. Forman en
filas apretadas y son muy efectivos en el combate cuerpo a cuerpo.

\perfil{Legión}{\atributos{50}{5}{0}{4}{4}{2}{1}{5}{7}}{\caracteristicas{1}{7}{0}{0}{10}{40}}{10}{10}{25}

\subsubsection*{Arcabuceros}
Los arcabuceros son los encargados de traer la tecnología a un mundo
dominado por las bestias. Poseen armas de pólvora cuyos disparos
penetran las mas duras armaduras y desgarran las mas escamosas o
gruesas pieles. Son torpes en el combate cuerpo a cuerpo y es
aconsejable que disparen desde una buena posición en algún lugar
alejado del campo de batalla.

\perfil{Arcabuceros}{\atributos{50}{4}{5}{3}{3}{1}{1}{1}{7}}{\caracteristicas{1}{4}{6}{200}{5}{20}}{10}{10}{25}

\subsubsection*{Ballesteros}
Los ballesteros son unidades de apoyo, menos precisos que los
arcabuceros, pero mas baratos, numerosos y con sus ballestas tienen un
mayor alcance. Pueden constituir una buena unidad para ser usada para el
desgaste inicial de los combates.

\perfil{Ballesteros}{\atributos{50}{4}{5}{3}{3}{1}{1}{1}{7}}{\caracteristicas{1}{4}{4}{300}{10}{30}}{10}{10}{20}

\subsubsection*{Escuderos}
Constituyen la única caballería ligera del ejército humano. Poseen
arcos cortos para disparar en sus travesías, y constituyen a su vez,
una unidad de desgaste inicial, y un apollo secundario en combates
cuerpo a cuerpo.

\perfil{Escuderos}{\atributos{125}{4}{5}{4}{3}{2}{1}{4}{7}}{\caracteristicas{2}{4}{4}{100}{5}{20}}{10}{20}{40}

\subsubsection*{Equites}
Constituyen una de las caballerías pesadas del ejército humano. Su
carga es dura, su presencia en combate imponente, y su firmeza excesiva.

\perfil{Equites}{\atributos{125}{5}{0}{6}{4}{3}{1}{7}{8}}{\caracteristicas{2}{6}{0}{0}{5}{20}}{10}{20}{65}

\subsubsection*{Unicornios}
Un unicornio es un caballo blanco con un cuerno largo y puntiagudo en
su frente. Son una caballería a su vez pesada por su fuerza y ligera por su velocidad,
lo que contituye una paradoja del ejército humano. Deben usarse con
cuidado, ya que, por su rareza, es una unidad muy cara de conseguir y
su pérdida puede suponer la derrota.

\perfil{Unicornios}{\atributos{150}{6}{0}{5}{4}{2}{1}{8}{9}}{\caracteristicas{2}{8}{0}{0}{5}{20}}{10}{20}{80}

\subsubsection*{Pegasos}
Es una caballería alada que posee el ejército humano. Pueden tener
todo el campo de batalla bajo su estricta vigilizancia, colocándose en
cualquier lugar del campo de batalla en el menor tiempo posible, para
hacer gala de su potencia en combate.

\perfil{Pegasos}{\atributos{200}{6}{0}{7}{5}{4}{2}{7}{8}}{\caracteristicas{2}{7}{0}{0}{3}{10}}{20}{20}{90}

\subsubsection*{Mantícora}
Una mantícora es un león rojizo con pequeñas alas y cabeza humana. Actúan solos
y son muy agresivos. Unen la potencia y la brutalidad de una bestia con
la disciplina y la inteligencia de un hombre, haciendo de ellos una
criatura mortal y horrenda para sus enemigos.

\perfil{Mantícora}{\atributos{100}{6}{0}{6}{5}{4}{6}{5}{10}}{\caracteristicas{4}{10}{0}{0}{1}{1}}{20}{20}{180}

\subsubsection*{Dragón}
Si un dragón hace presencia en un campo de batalla, el enemigo no
debería proseguir su campaña. Solo son propiedad de reyes y altos
caudillos, que en lejanos viajes y despues de tremendas guerras y
aventuras de honor se han hecho dueños y domadores de estos seres de
fantasía. Son muy fuertes, vuelan, escupen fuego, son
indesmoralizables y tienen gran resistencia al cansancio y la fatiga.

\perfil{Dragón}{\atributos{220}{7}{7}{7}{7}{6}{5}{7}{9}}{\caracteristicas{5}{10}{7}{150}{1}{1}}{50}{50}{300}

\subsection*{Orcos}
Los orcos son salvajes, indisciplinados, brutos, fieros, sucios y
tan torpes que casi ni siquiera saben usar el arco y la flecha, y
siempre vienen acompañados de criaturas todavía mas salvajes, mas
indisciplinadas, mas brutas, mas fieras y mas sucias aún si cabe,
constituyendo una horrible masa verde y marrón de olor pestilente que
provocan dolor y caos por donde quiera que pasen.

\subsubsection*{Guerreros}
Los guerreros orcos son la unidad mas básica de un ejército orco. Son
muy baratos y forman grandes unidades con una gran cantidad de
efectivos, pero son muy indisciplinados lo que los hacen muy poco
fiables en la batalla.

\perfil{Guerreros}{\atributos{50}{3}{0}{4}{4}{1}{1}{2}{6}}{\caracteristicas{1}{2}{0}{0}{20}{60}}{10}{10}{10}

\subsubsection*{Trasgos}
Los tragos son pequeñas criaturas traviesas aunque igual de indecentes
y asquerosas que los orcos. Pero aunque sean crédulos y poco
inteligentes, también son salvajes, peligros y con gran afición a la
sangre.

Al igual que los guerreros orcos, se mueven en gran multitud, aparecen
en cualquier rincón y lugar, y, aunque no suele salir a la luz del
sol, salen de sus madrigeras y forman multitud en cuanto el sol cae
por el horizonte.

\perfil{Trasgos}{\atributos{60}{3}{0}{3}{3}{1}{1}{2}{5}}{\caracteristicas{1}{1}{0}{0}{20}{100}}{8}{8}{5}

\subsubsection*{Onis}
Los onis son como los hermanos mayores de los orcos. Aunque no
sean la misma criatura ni tengan ninguna relación con ellos, tienen un
aspecto semejante e igual de bárbaro, comparten las mismas aficiones
bélicas, y es por ello que gustan de ir junto a los orcos.

Son generalmente mas grandes y fuertes que los orcos, y también son
algo mas disciplinados, lo que los hace muy peligrosos en el combate
cuerpo a cuerpo.

\perfil{Onis}{\atributos{50}{4}{0}{5}{4}{2}{1}{4}{7}}{\caracteristicas{2}{5}{0}{0}{20}{50}}{12}{12}{15}

\subsubsection*{Wargos}
Actúan como la caballería ligera de un ejército orco, aunque ellos no lo
sepan. Un wargo es una especie de lobo pero mas grande, feo y
salvaje, y en un ejército orco, son montados por guerreros orcos. El
wargo es un animal muy veloz, y sus montadores suelen llevar un arco,
aunque no sean muy hábiles en su uso, para disparar a todo lo que le
rodea mientras eligen quién será su próximo enemigo.

\perfil{Wargos}{\atributos{125}{3}{5}{4}{4}{2}{1}{2}{6}}{\caracteristicas{2}{4}{5}{160}{10}{25}}{10}{20}{30}

\subsubsection*{Jabalíes}
Un jabalí es una criatura mas grande que un wargo, con colmillos mas
largos y un hocico baboso. Los orcos también tienen colmillos y
babean. Los jabalíes acostumbran a revolverse entre el fango. Los
orcos también tienen bastante fango en su cuerpo. Ahora coloca a tal
pareja variopinta de criaturas, la primera subida a lomos de la
segunda. Así es la caballería orca.

\perfil{Jabalíes}{\atributos{90}{4}{0}{5}{4}{3}{1}{3}{7}}{\caracteristicas{2}{6}{0}{0}{10}{20}}{15}{20}{30}

\subsubsection*{Garms}
Los garms no son tan sucios como los jabalíes a los que acostumbran a
montar los orcos, pero sí que son grandes, feos y fieros. Poseen una
fuera descomunal y son muy rudos, lo que hace a las cargas de una
unidad de garms uno de los fenómenos mas brutales que puedan suceder
en una batalla.

\perfil{Garms}{\atributos{100}{4}{0}{6}{5}{4}{2}{2}{7}}{\caracteristicas{3}{7}{0}{0}{5}{15}}{20}{40}{50}

\subsubsection*{Trolls}
Un troll es una criatura grande de aspecto humanoide y deforme, de
piel algo escamosa y cubierta de lodo, lo que lo hace una criatura muy
resistente a los golpes. No son muy habilidosos, pero sus golpes son
muy potentes, y acostumbran a llevar objetos con lo que aplastar a sus
enemigos. No es muy inteligente, y no sabe llevar a cabo ningún tipo
de táctica ni estrategia, pero su estupidez y su brutalidad lo hacen
idóneos para mandarlas a combatir contra unidades rápidas o dañinas
del enemigo, para mantenerlas ocupadas mientras se desenvuelve el
resto de la batalla.

\perfil{Trolls}{\atributos{75}{3}{0}{5}{5}{3}{3}{7}{8}}{\caracteristicas{2}{7}{0}{0}{3}{10}}{30}{30}{80}

\subsubsection*{Abominación}
Una abominación es una criatura que deambula sola provocando el caos a
su paso. Es un gran agente de enfermedades (a la que los orcos ya
están acostumbrados) y da pavor enfrentarse a ellas. Es muy dificil
acertar al efectuarle golpes y los que impacten tampoco harán mucho
efecto en él.

\perfil{Abominación}{\atributos{75}{5}{0}{6}{6}{4}{4}{2}{10}}{\caracteristicas{4}{8}{0}{0}{1}{1}}{50}{50}{90}

\subsubsection*{Gigante}
Los gigantes son una antigua raza de hombres, excluidas de la
sociedad, que con el paso del tiempo se fueron embruteciendo,
aumentando de tamaño y perdiendo progresivamente su inteligencia, lo
que los ha convertido en unos seres primitivos pero con gran capacidad
destructiva, motivo que los ha hecho establecer contacto con la raza
de los orcos.

\perfil{Gigante}{\atributos{75}{5}{0}{7}{6}{6}{5}{1}{8}}{\caracteristicas{5}{7}{0}{0}{1}{1}}{50}{50}{150}