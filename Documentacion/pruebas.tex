% Este archivo es parte de la memoria del proyecto fin de carrera
% de Aarón Bueno Villares. Protegida bajo la licencia GFDL.
%
% Para más información, la licencia completa viene incluida en el
% fichero fdl-1.3.tex

% Copyright (C) 2010 Aarón Bueno Villares

\section{Pruebas}
\label{sec:pruebas}

Para diseñar los casos de prueba, distinguimos entre aquellas clases y
estructuras que, por su naturaleza, pueden ser testeadas
individualmente, de aquellas que se testean de forma integrada. La
mayoría de las funciones ofrecidas por el gestor de escenario fueron
testeadas de forma individual, al igual que todas las
estructuras matemáticas. En cambio, el resto de clases y funciones,
por su simplicidad y por su dependencia gráfica, no merecía la pena
realizar casos de prueba individuales, quizás mas complejos que
estudiar, simplemente, su comportamiento integrado. Así que, en
definitiva, la organización de los casos de prueba fue la siguiente:

\begin{enumerate}
\item A medida que se fueron implementando, se probaron
  individualmente todas las estructuras y funciones matemáticas
  diseñadas y las funciones de corte mas matemático.
\item Una vez superados los test anteriores, y a medida que se
  agregaba nueva funcionalidad, se probaba cada vez el sistema
  completo de forma integrada.
\item Al finalizar todo el proceso completo de diseño e implementación
  comenzaron las pruebas beta. Algunos usuarios probaron el producto y
  no se detectó ningún defecto adicional.
\end{enumerate}

\subsection{Pruebas alfa}
Las pruebas alfa son las pruebas que se realizan durante el proceso de
desarrollo. En el caso que nos ocupa, las pruebas se realizaban al
final de cada iteración de implementación.

\subsubsection{Pruebas unitarias}
Las funciones sustentas a este tipo de pruebas, como hemos dicho
antes, eran las mas relacionadas con el ámbito mas matemático del
juego. Debido a su complejidad, las pruebas se realizaron bajo un
enfoque estructural, es decir, de caja blanca, y además, bajo un
criterio de cobertura de sentencias, es decir, se estableció que la
mejor forma de estudiar el comportamiento correcto de una función era
lograr que al menos cada posible rama de la función fuese explorada,
con la finalidad de que al menos cada sentencia fuera ejecutada una
vez. A su vez, se diseñaron las pruebas de modo que las sentencias mas
problemáticas de cada función (aquellas que poseen divisiones,
peligros de desbordamiento, bucles o ecuaciones complejas) se probaran
varias veces explorando tanto los casos mas comunes como los mas
extremos (análisis de valores límite), de modo que se examine de forma
exhaustiva su comportamiento.

Como era de esperar, el mayor núcleo de errores de esta clase se
encontraban en la implementación de las estructuras matemáticas del
juego, que tenían alta implicación en el comportamiento del gestor de
escenarios.

\subsubsection{Pruebas de integración}
Las pruebas de integración del sistema complejo al completo (en cada
iteracción de diseño), debida a la relativa sencillez estructural de
las clases que la conformaban (una funcionalidad compleja pero formada
por un rico dinamismo de funciones individuales muy sencillas), se
realizaron bajo un enfoque funcional o de caja negra. Se probaba,
sencillamente, que la ejecución de la nueva funcionalidad
implementada, en consonancia con funcionalidades completadas previas,
realizara con éxito lo que se supone que debía realizar.

El principal núcleo de defectos encontrados bajo estas pruebas de
integración recayeron sobre el gestor de iconos, sobre el gestor de
ejércitos, y en los procesos de destrucción (cuando finaliza una
partida). 

\subsection{Pruebas beta}
Al finalizar el proceso de diseño, implementación y pruebas alfa se
ofreció el juego a distintos usuarios (generalmente compañeros de
universidad) para que probaran el juego y me remitieran los errores
producidos y el contexto del mismo. Luego repetía las situaciones
reportadas, hasta obtener el error nuevamente y explorar su origen,
corregir y reenviar la nueva versión a mis
\emph{ayudantes}. Generalmente, los fallos y defectos encontrados en
esta fase de pruebas fueron relacionadas con el proceso de huida de
las unidades desmoralizadas en combate.
