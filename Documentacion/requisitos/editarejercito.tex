\reqFuncional{Editar ejército}{Modifica un ejército existente}{Un usuario y el sistema}{El usuario se encuentra frente a la pantalla principal}
{
  \item Desde la pantalla principal, el usuario indica que desea modificar un ejército.
  \item Se muestra una pantalla con la lista de ejércitos actuales.
  \item El usuario elige un ejército. \label{meini}
  \item El usuario indica que desea modificarlo. \label{memodificar}
  \item <<include>> \ref{Modificar ejército (caso de uso abstracto)}: Modificar ejército. \label{meeditar}
  \item El usuario ordena guardar los cambios. \label{mefin}
  \item El sistema guarda los cambios del ejército. \label{meguardar}
  \item El caso de uso finaliza con éxito.
}
{
  \item[\ref{meini}-\ref{mefin}:] El usuario cancela la operación.
    \begin{enumerate}
    \item El caso de uso finaliza sin éxito.
    \item Se muestra de nuevo la pantalla principal.
    \end{enumerate}
  \item[\ref{memodificar}:] El usuario elimina el ejército.
    \begin{enumerate}
    \item Vuelve al paso \ref{meini}.
    \end{enumerate}
  \item[\ref{mefin}:] El usuario desea cambiar el nombre de su ejército.
    \begin{enumerate}
    \item El usuario indica el nuevo nombre que dará a su ejército. El nombre no deberá tener más de 30 carácters de longitud.\label{meinuevonombre} 
      \begin{enumerate}
      \item El usuario cancela la operación.
      \item Vuelve al paso \ref{meeditar}.
      \end{enumerate}
    \item El sistema comprueba que no exista un ejército con el mismo nombre.
      \begin{enumerate}
      \item Ya existe un ejército con el mismo nombre.
      \item Vuelve al paso \ref{meinuevonombre}.
      \end{enumerate}
    \item Continúa en el paso \ref{meguardar}.
    \end{enumerate}
}{Se actualiza el ejército satisfactoriamente.}