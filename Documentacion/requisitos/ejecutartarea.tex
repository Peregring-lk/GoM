\reqFuncional{Ejecutar tarea (caso de uso abstracto)}{Realiza la acción asociada a una tarea}{Dos usuarios y el sistema}{Dos usuarios están jugando una partida y el sistema ha mostrado el icono asociado a la tarea}
{
  \item El usuario cuyo turno está en juego desea realizar una acción indicándolo con el icono de la tarea.
  \item El sistema comprueba que la tarea esté permitida en el contexto actual de la partida, según el reglamento de \textit{GoM}. \label{etini}
  \item Si son necesarios, el sistema espera a que el usuario introduzca interactivamente los datos asociados a la acción que desea realizar (cantidad de movimiento, objetivo de una carga, etc).
  \item El sistema comprueba que los datos introducidos sean correctos. \label{etdatos}
  \item El sistema ejecuta la acción y se reflejan en la pantalla los cambios producidos en el juego.\label{etejecuta}
  \item El caso de uso finaliza con éxito. \label{etfin}
}
{
  \item[\ref{etini}-\ref{etdatos}:] El usuario cancela la acción.
    \begin{enumerate}
    \item No se hace nada y el caso de uso finaliza sin éxito.
    \end{enumerate}
  \item[\ref{etdatos}:] Los datos introducidos no son correctos.
    \begin{enumerate}
    \item Se muestra la descripción del error en el panel de errores.
    \item El caso de uso finaliza sin éxito.
    \end{enumerate}
  \item[\ref{etejecuta}:] La acción provoca el fin de la partida.
    \begin{enumerate}
    \item El sistema manda una señal de fin de partida.
\item Vuelve al paso \ref{etfin}.
    \end{enumerate}
}
{Se realiza la acción asociada a la tarea elegida.}
