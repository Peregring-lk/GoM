\reqFuncional{Elegir raza (caso de uso abstracto)}{El usuario desea cambiar la raza de su ejército}{Un usuario y el sistema}{El usuario se encuentra editando un ejército}
{
  \item El usuario indica que desea elegir una nueva raza para su ejército, seleccionando una de la lista de razas disponibles.
  \item Si existen, se borran todas las unidades anteriormente añadidas.
  \item El caso de uso finaliza con éxito.
}
{
  \item Ninguno.
}
{El ejército es de la nueva raza elegida.}
