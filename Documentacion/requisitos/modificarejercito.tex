\reqFuncional{Modificar ejército (caso de uso abstracto)}{Se edita un ejército}{Un usuario y el sistema}{El usuario ha elegido editar un ejército}
{
  \item El ejército a editar ya existe. \label{eeexiste}
  \item Se muestra una pantalla con la configuración actual del ejército: raza actual, unidades del ejército, y posición, frente y número de efectivos de cada unidad.
  \item El usuario, sin cambiar de raza, desea modificar su ejército. \label{eeraza}
  \item El usuario puede añadir nuevas unidades a su ejército o eliminar las existentes. \label{eeini}
  \item El usuario puede eliminar unidades existentes de su ejército.
  \item El usuario puede modificar posición, frente o número de efectivos de cada unidad. \textit{GoM}. \label{eemoduni}
  \item El usuario indica que ha terminado de editar su ejército. \label{eefin}
  \item El sistema comprueba que el ejército actual cumple con la reglas impuestas por el reglamento de \textit{GoM}. \label{eereglas}
  \item El caso de uso finaliza con éxito.
}
{
  \item[\ref{eeexiste}:] El ejército a editar es nuevo.
    \begin{enumerate}
    \item <<extern>> \ref{Elegir raza (caso de uso abstracto)}: Elegir raza.
    \item Va al paso \ref{eeraza}.
    \end{enumerate}
  \item[\ref{eeraza}-\ref{eefin}:] El usuario cancela la operación.
    \begin{enumerate}
    \item Se deshacen los cambios realizados.
    \item Se vuelve al caso de uso anterior.
    \item El caso de uso finaliza sin éxito.
    \end{enumerate}
  \item[\ref{eeini}-\ref{eemoduni}:] El usuario desea cambiar de raza.
    \begin{enumerate}
    \item <<extern>> \ref{Elegir raza (caso de uso abstracto)}: Elegir raza.
    \item Va al paso \ref{eeraza}.
    \end{enumerate}
  \item[\ref{eereglas}:] La configuración actual del ejército no es permitida por el reglamento de \textit{GoM}.
    \begin{enumerate}
    \item Se indican los errores cometidos.
    \item Se vuelve al paso \ref{eeini}.
    \end{enumerate}
}{El sistema guarda los cambios del nuevo ejército}